\documentclass{article}
\usepackage{cancel}
\usepackage[utf8]{inputenc}
\usepackage {titlesec}
\usepackage[english,russian]{babel}
\usepackage{graphicx}
\usepackage{amssymb}
\graphicspath{{pictures/}}
\usepackage{amsmath}
\titlespacing*{\section}{\parindent}{*4}{*4}
	
	\title{Домашнее задание 7}
	\author{Ткачев Андрей, группа 166}
	\date{\today}

\begin{document}
	\maketitle
	
	\section{Задача 1}
	
	\subsection{1}
	
	 Если $c|a$ и $c\nmid b$, то $a = ck_0$, $b = ck_1 + r$. Тогда $a + b = c(k_0 + k_1) + r$. $\Rightarrow$ $c \nmid (a + b)$.
	 
	 Утверждение $(1)$ верно.
	
	\subsection{2}
	
	$ 7 $ не делится на $ 8 $ и $ 1 $ не делится на $ 8 $, но $ 1 + 7 $ кратно $ 8 $.
	
	Утверждение $(2)$ неверно.
	
	\subsection{3}
	
	14 не делится на 8 и 12 не делится на 8, но $8 \mid 12 \cdot 14$.
	
	Утверждение $(3)$ не верно.
	
	\subsection{4}
	
	Если $c\mid a$ и $c\mid b$, то $a = ck_0$, $b = ck_1$. Тогда
	$ab = c^2k_0k_1$.
	
	Значит $c^2\mid ab$. Утверждение $(4)$ верно.
	
	\section{Задача 2}
	
	\subsection{А}

	Поймем, что $2^{15} | 20!$, более того $2^{18} | 20!$ Выпишем все четные множители, входящие в $20!$:
	
	$$2, 4, 6, 8, 10, 12, 14, 16, 18, 20$$
	
	Посчитаем степень суммарную степень двойки, входящую в произведение:
	
	$$1 + 2 + 1 + 3 + 1 + 2 + 1 + 4 + 1 + 2 = 18$$
	
	Т.е. $20! = 2^{18} \cdot k$  $\Rightarrow 20! \equiv 0 \mod 2^{15}$.
	
	\subsection{Б}
	
	Как мы еще помним, $2^{18} | 20!$, а значит $20! = 2^{18} \cdot k$, причем $k = 2m + 1$, т.е. 2 входит в $20!$ в степени 18. Значит $20! = 2^{18} (2m + 1) = 2^{19}m + 2^{18}$.
	
	Тогда $20! \equiv 2^{18} \mod 2^{19}$.
	
	\section{Задача 3}
	
	$x^2 \equiv 1 (\mod 2^n) \Leftrightarrow x^2 - 1 \equiv 0 (\mod 2^n)$
	
	Тогда $(x - 1)(x + 1) \equiv 0 \mod 2^n$. Заменим $x - 1 = y$. Перепишем равенство:
	
	$$ y(y + 2) \equiv 0 \mod 2^n  \textbf{(1)}$$
	
	Нас интересуют те y, для которых верно $0 \leqslant y + 1 < 2^n $ \textbf{(2)}.
	Решениями очевидно являются $y = 0,\space y = 2^n - 2$, иначе, раз имеет место быть сравнение (1), $y = a2^k,\space y + 2 = b2^m$, причем $2^n \leqslant m + k,\space 2\nmid a,b$ и $a,b > 0$.Но 
	$$y + 2 = a2^k + 2 = 2(a2^{k-1} + 1) = b2^m$$
	$$a2^{k - 1} + 1 = b2^{m - 1}$$ 
	
	Значит либо $k = 1$, либо $m = 1$. В первом случае $y + 2 = 2^{n - 1}; y = 2^{n - 1} - 2$. Во втором: $y=2^{n - 1}$. (Поймем, что  $y \ne c2^{n - 1} (- 2)$, при $c > 1$ т.к. мы должны быть в рамках неравенства (2))
	
	Тогда получаем 4 серии решений $y$. Перейдем от них к $x$.
	
	\textbf{Ответ:} $$1$$
	$$2^{n - 1} \pm 1$$
	$$2^n - 1$$
	
	При $n=2$ две пары решений эквивалентны.
	
	\section{Задача 4}
	
	$\gcd(2^{2016} - 1, 2^{450} - 1) = \gcd(2^{2016} - 2^{450}, 2^{450} - 1) = \gcd(2^{450}(2^{1556} - 1), 2^{450} - 1)$
	
	Т.к. $\gcd(2^{450}, 2^{450} - 1) = 1$, 
	то
	$$\gcd(2^{450}(2^{1556} - 1), 2^{450} - 1) = \gcd(2^{1556} - 1, 2^{450} - 1)$$
	
	$$\gcd(2^{450}(2^{1116} - 1), 2^{450} - 1) = \gcd(2^{1116} - 1, 2^{450} - 1)$$
	
	$$\gcd(2^{450}(2^{666} - 1), 2^{450} - 1) = \gcd(2^{666} - 1, 2^{450} - 1)$$
	
	$$\gcd(2^{450}(2^{216} - 1), 2^{450} - 1) = \gcd(2^{216} - 1, 2^{450} - 1)$$
	
	$$\gcd(2^{216}(2^{234} - 1), 2^{216} - 1) = \gcd(2^{216} - 1, 2^{234} - 1)$$
	
	$$\gcd(2^{216}(2^{18} - 1), 2^{216} - 1) = \gcd(2^{216} - 1, 2^{18} - 1)$$
	
	Поймем, что $2^{216} - 1 = (2^{108} - 1)(2^{108} + 1) = (2^{54} - 1)(2^{54} + 1)(2^{108} + 1) = (2^{27} - 1)(2^{27} + 1) \cdot ...$
	
	Но $$2^{27} - 1 = (2^9 - 1)(2^{18} + 2^9 + 1),$$ $$2^{27} + 1 = (2^9 + 1)(2^{18} - 2^9 + 1).$$ 
	
	Тогда $2^{216} - 1 = (2^9 - 1)(2^9 + 1) \cdot ... = (2^{18} - 1)\cdot ...$
	
	$$\gcd(2^{216} - 1, 2^{18} - 1) = 2^{18} - 1.$$
	
	\textbf{Ответ:} $2^{18} - 1.$
	
	\section{Задача 5}
	
	$74x \equiv 1 \mod 47$
	
	Поймем, что такой $x$ существует, т.к. $\gcd(74, 47) = 1$.
	Найдем $x$ решив уравнение $74x - 47y = 1$, пользуясь алгоритмом Евклида:
	$$a_0 = 74 \cdot 1 - 47 \cdot 0$$  
	$$a_1 = 74 \cdot 0 + 47 \cdot 1  $$ 
	$$a_{i - 2} = 74x_{i - 2} + 47y_{i - 2}$$ 
	$$a_{i - 1} = 74x_{i - 1} + 47y_{i - 1}$$ 
	$$a_i = a_{i - 2} \% a_{i - 1} = 
	a_{i - 2} - a_{i - 1}\lfloor\frac{a_{i-2}}{a_{i - 1}}\rfloor =
	74(x_{i - 2 - \lfloor\frac{a_{i-2}}{a_{i - 1}}\rfloor }x_{i - 1}) + 47(y_{i - 2 - \lfloor\frac{a_{i-2}}{a_{i - 1}}\rfloor }y_{i - 1})$$  
	
	$$a_2 = 74 \cdot 1 - 47\cdot1 = 27$$
	$$a_3 = -74 \cdot 1 + 47\cdot2 = 20$$
	$$a_4 = 74 \cdot 2 - 47\cdot3 = 7$$
	$$a_5 = -74 \cdot 5 + 47\cdot 8 = 6$$
	$$a_6 = 74 \cdot 7 - 47\cdot 11 = 1$$
	
	Таким образом искомый вычет $7$.
	
	\textbf{Ответ:} 7.
	
	\section{Задача 6}
	
	Количество решений сравнения $39x \equiv 104 \mod 221$ равно количеству решений сравнения $3x \equiv 8 \mod 17$ умноженному на 13 (т.к. каждое решение нового сравнения меньше 17, но при этом, если к нему прибавить $17k$, то получим решение начального сравнения, $0< k <13$), т.к. $\gcd(39, 104, 221) = 13$.
	\\
	$$3x \equiv 8 \mod 17$$
	
	Т.к. 3 взаимно просто с 17, то можно обратить 3 по модулю.
	
	$$3^{-1} \equiv 6 \mod 17$$
	
	Домножим исходное на 6, получаем:
	
	$$ x \equiv 14 \mod 17$$
	
	Это эквивалентно 13 решениям по модулю $17 \cdot 13$: $$13, 13 + 17, 13 + 17\cdot 2, ..., 13 + 17\cdot 12.$$
	
	\textbf{Ответ:} 13 решений. 
                             
	\section{Задача 7}                                                                                                                                                   
	
	Число делится на 22, если оно делится на 2 и на 11. Тогда $22 | n^{10} - 1$, если $n$ нечетно и $11|n^{10} - 1$. Очевидно, что $n$ не должно делится на 11. Т.к. $11$ - простое число, то по м. теореме Ферма: $n^{10} \equiv n \mod 11$ $\Rightarrow$ $n \equiv 1 \mod 11$.

	
	\textbf{Ответ:} для всех нечетных $n$, не делящихся на 11
	
	\section{Задача 8}
	
	Посмотрим внимательно на сумму:
	
	$$ 1 + \frac{1}{2} + ... + \frac{1}{p - 2} + \frac{1}{p - 1} $$
	
	Поймем про нее две вещи:
	\begin{itemize}
		\item Количество слагаемых четно
		
		\item Сумма слагаемых $i$ и $p - i$ имеет вид: $\frac{p}{(p - i)i}$
\end{itemize}	
	Тогда разобьем сумму на такие парные слагаемые, раз всего слагаемых четное число. Получим сумму дробей, знаменатель которых $p$. Вынесем $p$, как общий множитель и сложим дроби $\frac{1}{1(p - 1)} + \frac{1}{2(p - 2)} + ...  $, выполнив сокращение до несократимой, и умножим числитель на оставшуюся вне скобок $p$. Поймем, что мы получили несократимую дробь ($p$ не делится ни одно из чисел $1 .. p-1$), которая равна начальной, а числитель ее кратен $p$, что мы и хотели доказать.
	
	
\end{document}