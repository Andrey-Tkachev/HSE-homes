\documentclass{article}
\usepackage{cancel}
\usepackage[utf8]{inputenc}
\usepackage {titlesec}
\usepackage[english,russian]{babel}
\usepackage{amssymb}
\usepackage{amsmath}
\titlespacing*{\section}{\parindent}{*4}{*4}

\title{Домашнее задание 8}
\author{Ткачев Андрей, группа 166}
\date{\today}

\begin{document}
	\maketitle
	\section {Задача 1}
	Разложим 2007 на простые множители: $2007 = 223 \cdot 3 \cdot 3$. По условию задачи произведение чисел, образованных оценками выписанными подряд и разделенными знаками умножения равно 2007. Тогда в силу того, что обычно школьная оценка - число от 2 до 5 (колы не ставятся по условию) и что разложение на простые множители существует и единственно, Петя должен был выписать оценки <<2, 2, 3>> (обязательно строго подряд) и еще как-то оставшиеся две тройки. Тогда оценка в четверти у Пети выходит не очень хорошая, а именно (в допущении, что оценка за четверть - просто среднее арифметическое) $2.6$.
	
	\textbf{Ответ:} $2.6$.
	
	\section {Задача 2}
	
	\subsection {а)} Рассмотрим какие-нибудь последовательные числа $b_1, b_2, ... b_n$ ($b_i < b_j \Leftrightarrow i < j$). 
	
	Если $n | b_1$ - отлично произведение этих $n$ чисел делится на $n$.
	
	 Иначе, пусть $b_1 \equiv r \mod n$. Тогда $b_2 \equiv r + 1 \mod n$ и т.д. Т.е. $b_i \equiv r + i - 1 \mod n$ $\forall i \leqslant n$. Рассмотрим тогда $b_{n - r + 1}$, $n - r + 1 \leqslant n$.
	 $b_{n - r + 1} \equiv r + (n - r + 1) - 1 = n \equiv 0 \mod n$. Т.е. в произведение $n$ последовательных чисел $n - r + 1$ множитель (нумерация в порядке возрастания), где $r$ - ненулевой остаток первого из чисел (т.е. наименьшего из чисел), кратен $n$ $\Rightarrow$ произведение $b_1 \cdot ... \cdot b_{n - r + 1}  \cdot ... \cdot b_n$ делится на $n$.
	 
	 \subsection {б)} Докажем, что $\forall n, k\in N$ число $\frac{\prod_{i=0}^{n - 1} k + i}{n!} \in Z$.
	 Для этого докажем тождество $ {k + n - 1 \choose k - 1} =  \frac{\prod_{i=0}^{n - 1} k + i}{n!}$ :
	 
	 $${k + n - 1 \choose k - 1} = \frac{(k + n - 1)!}{(k - 1)!n!} = \frac{k(k + 1) \cdot ... \cdot (k + n - 1)}{n!} = \frac{\prod_{i=0}^{n - 1} k + i}{n!}$$
	 
	 Но число сочетаний ${k + n - 1 \choose k - 1} \in Z \Rightarrow \frac{\prod_{i=0}^{n - 1} k + i}{n!} \in Z$	
	 $\Rightarrow $ $n! | \prod_{i=0}^{n - 1} k + i$, для всех $n, k \in N$, что и требовалось доказать.
	
	\section{Задача 3}
		Воспользуемся расширенным алгоритмом Евклида, чтобы решить уравнение $45x^\prime - 37y^\prime = 1$:
		
		$$ a_0 = 1 \cdot 45 + 0 \cdot 37 $$ 
		$$ a_1 = 0 \cdot 45 + 1 \cdot 37 $$ 
		$$ a_2 = 0 \cdot 45 + 1 \cdot 37 - (1 \cdot 45 + 0 \cdot 37) \cdot \lfloor\frac{45} {37}\rfloor = 1 \cdot 45 - 1 \cdot 37 = 8 $$
		$$ a_3 = 1 \cdot 45 - 1 \cdot 37 - (0 \cdot 45 + 1 \cdot 37) \cdot \lfloor\frac{37} {8}\rfloor = -4 \cdot 45 + 5 \cdot 37 = 5 $$
		$$ a_4 = -4 \cdot 45 + 5 \cdot 37 - (1 \cdot 45 - 1 \cdot 37) \cdot \lfloor\frac{8} {5}\rfloor = 5 \cdot 45 - 6 \cdot 37 = 3 $$
		$$ a_5 = 5 \cdot 45 - 6 \cdot 37 - (-4 \cdot 45 + 5 \cdot 37) \cdot \lfloor\frac{5} {3}\rfloor = -9 \cdot 45 + 11 \cdot 37 = 2 $$
		$$ a_6 = -9 \cdot 45 + 11 \cdot 37 - (5 \cdot 45 - 6 \cdot 37) \cdot \lfloor\frac{3} {2}\rfloor = 14 \cdot 45 - 17 \cdot 37 = 1 $$
		
		Получаем $x^\prime = 14$, $y^\prime = 17$. Покажем, что все остальные решения в целых числах имеют вид $x^{\prime\prime} = 14 + 37k$, $y^{\prime\prime} = 17 + 45k$, $k \in Z$.
		Пусть $(x_0, y_0)$ и $(14, 17)$ - два различных решения уравнения $45x^\prime - 37y^\prime = 1$. Тогда:
		
		$$ 45\cdot 14 - 37\cdot 17 = 1$$ 
		$$ 45x_0 - 37y_0 = 1$$
		$$ \Downarrow $$
		$$ 45 (14 - x_0) = 37 (17 - y_0) $$
		
		Пусть $x_0 = 14 + n$, $y_0 = 17 + m$.
		
		$$ 45 n = 37 m $$
		
		Т.к. $ (45, 37) = 1$, то $37 | n$ и $45 | m$. Тогда $m = 45k$, т.к. по ОТА разбиение на простые множители существует и единственно, то $3:2 \cdot 5$ содержатся в множителях $m$.
		
		$$ 45 n = 37 \cdot 45k$$
		$$ n = 37k$$
		
		От куда и получаем $x^{\prime\prime} = 14 + 37k$, $y^{\prime\prime} = 17 + 45k$.
		Тогда решения уравнения $45x - 37y = 25$: $(25(14 + 37k), 25(17 + 45k))$.
		
		\textbf{Ответ:} $(25(14 + 37k), 25(17 + 45k))$, где $k \in Z$.	
		
		\section {Задача 4}
		
		\subsection {a)} $111...111 = \frac{10^{69} - 1}{9}$. Решим уравнение:
		$$111...111 = \frac{10^{69} - 1}{9} \equiv x \mod 71$$ 
				$$10^{69} - 1 \equiv 9x \mod 71$$
				$$10^{70} - 10 \equiv 90x \mod 71$$
		
		По малой т. Ферма $10^{70} \equiv 1 \mod 71$.
		
				$$10^{70} - 10 \equiv 90x \mod 71$$
				$$1 \equiv 90x + 10 \mod 71$$
				$$-9 \equiv 90x \mod 71$$
				$$62 \equiv 90x \mod 71$$
				$$31 \equiv 45x \mod 71$$
				
		Заметим, что $ 45^{-1} \equiv 30 \mod 71$ (убедится в этом можно воспользовавшись расш. алгоритмом Евклида).
		
		$$ a_0 = 1 \cdot 71 + 0 \cdot 45 $$ 
		$$ a_1 = 0 \cdot 71 + 1 \cdot 45 $$ 
		$$ a_2 = 0 \cdot 71 + 1 \cdot 45 - (1 \cdot 71 + 0 \cdot 45) \cdot \lfloor\frac{71} {45}\rfloor = 1 \cdot 71 - 1 \cdot 45 = 26 $$
		$$ a_3 = 1 \cdot 71 - 1 \cdot 45 - (0 \cdot 71 + 1 \cdot 45) \cdot \lfloor\frac{45} {26}\rfloor = -1 \cdot 71 + 2 \cdot 45 = 19 $$
		$$ a_4 = -1 \cdot 71 + 2 \cdot 45 - (1 \cdot 71 - 1 \cdot 45) \cdot \lfloor\frac{26} {19}\rfloor = 2 \cdot 71 - 3 \cdot 45 = 7 $$
		$$ a_5 = 2 \cdot 71 - 3 \cdot 45 - (-1 \cdot 71 + 2 \cdot 45) \cdot \lfloor\frac{19} {7}\rfloor = -5 \cdot 71 + 8 \cdot 45 = 5 $$
		$$ a_6 = -5 \cdot 71 + 8 \cdot 45 - (2 \cdot 71 - 3 \cdot 45) \cdot \lfloor\frac{7} {5}\rfloor = 7 \cdot 71 - 11 \cdot 45 = 2 $$
		$$ a_7 = 7 \cdot 71 - 11 \cdot 45 - (-5 \cdot 71 + 8 \cdot 45) \cdot \lfloor\frac{5} {2}\rfloor = -19 \cdot 71 + 30 \cdot 45 = 1 $$
		
		Тогда домножим сравнение $31 \equiv 45x \mod 71$ на $30$.
			
			$$930 \equiv 45 \cdot (45^{-1})x \mod 71$$
			$$7 \equiv x \mod 71$$
			
		\textbf{Ответ:} 7.
		
		\subsection {б)} Согласно китайской теореме об остатках существует такой $x$, что 
			\begin{equation*} 
				\begin{cases}
					x \equiv 111...111 \mod 2 \\
					x \equiv 111...111 \mod 5 \\
					x \equiv 111...111 \mod 7
				\end{cases}
			\end{equation*}
		Причем  $x \equiv 111...111 \mod 2 \cdot 5 \cdot 7 $.
		
		Найдем тогда остатки $111...111$ от деления на 2, 5, 7:
			$$1...1 = 1...10 + 1 \equiv 1 \mod 2$$
			$$1...1 = 1...1 \cdot 2 \cdot 5 + 1 \equiv 1\mod 5$$
		
		Заметим, что число $11..11000$ (66 единиц), кратно $111111$, так как $11..11 = 111111 \cdot 10^{60} + ... + 111111 \cdot 10 ^ 6 + 111111$, где $11..11$ - 66 единиц.   
			$$1...1 = 11...11000 + 111 = 111111 \cdot k + 111  = 7 \cdot 15873  \cdot a + 111 \equiv 111 \equiv 6\mod 7$$
	    Тогда 
	    \begin{equation*} 
		    \begin{cases}
			    x \equiv 1 \mod 2 \\
			    x \equiv 1 \mod 5 \\
			    x \equiv 6 \mod 7
		    \end{cases}
	    \end{equation*}
		
		Поймем, что число 41 удовлетворяет сравнениям. Тогда $41 \equiv 111...111 \mod 2 \cdot 5 \cdot 7 $
		
		$$ 41 \equiv 111...111 \mod 70 $$
		
		\textbf{Ответ:} 41.
		
		\section{Задача 5}
		  
		  Посчитаем функцию Эйлера $\varphi(n)$. Пусть $n = p_0^{k_0} \cdot p_1^{k_1} \cdot ... \cdot p_m^{k_m}$. Тогда $$\varphi(n) = n \prod_{p|n}\frac{(p - 1)}{p}$$
		  
		  Заметим, что $\varphi(n) | n!$: если поделить $n!$ на $\varphi(n)$ получим  
		  $$\frac {n! \prod_{p|n}p} {\prod_{p|n} (p - 1)}$$
		  
		  Но т.к. $\forall p, p|n: p < n \Rightarrow 1 < p - 1 < n$. Но тогда каждое из чисел $p - 1$ входит в разбиение $n!$ на множители по определению (т. е. на в разбиение $n! = 1 \cdot ... \cdot$), значит $n!$ делится на произведение $\prod_{p|n}{p - 1}$  $\Rightarrow$ $\frac {n!}{\varphi(n)} = k \in N \Rightarrow \varphi(n) | n!$
		  
		  Тогда, т.к. $(n, 4) = 1$ - по условию $n$ нечетное положительное, то по теореме Эйлера:
		  $$ 4^{\varphi(n)} \equiv 1 \mod n $$
		  
		  Умножив это сравнение на себя $k = \frac {n!}{\varphi(n)} \in N$ раз получим:
		  $$ (4^{\varphi(n)})^{k} \equiv 1 \mod n $$
		  $$ 4^{\varphi(n) \cdot k} \equiv 1 \mod n $$
		  $$ 4^{n!} \equiv 1 \mod n $$
		  $$\Updownarrow$$
		  $$n | (4^{n!} - 1)$$		  
		  Что и требовалось доказать.
		  
		  \section {Задача 6}	
		  Обозначим количество книг за $n$. Тогда из условия следует, что $7 | n$ $\Rightarrow$ $n = 7k$, $k \in N$ (Т.к. n имеет не нулевые остатки при делении на 4, 6, 5, и 0-ой при делении на 7). Тогда $n = 7k$ является решением системы 
		  \begin{equation*} 
			  \begin{cases}
			  7k \equiv 1 \mod 4 \\
			  7k \equiv 1 \mod 5 \\
			  7k \equiv 1 \mod 6
			  \end{cases}
		  \end{equation*}
		  
		  т.к. при делении книг в группы по 4, 5, 6 остается одна лишняя. Зная, что
		  
		  \begin{equation*} 
		  \begin{cases}
		  7 \equiv 3 \mod 4 \\
		  7 \equiv 2 \mod 5 \\
		  7 \equiv 1 \mod 6
		  \end{cases}
		  \end{equation*}
		  
		  Имеем:
		  
		  \begin{equation*} 
		  \begin{cases}
		  3k \equiv 1 \mod 4 \\
		  2k \equiv 1 \mod 5 \\
		  k  \equiv 1 \mod 6
		  \end{cases}
		  \end{equation*}
		  
		  Домножив сравнения на обратные для чисел $3, 2, 1$ по модулям $4, 5, 6$ соответственно, получим
		  
		  \begin{equation*} 
		  \begin{cases}
		  k \equiv 3 \mod 4 \\
		  k \equiv 3 \mod 5 \\
		  k \equiv 1 \mod 6
		  \end{cases}
		  \end{equation*}
		  Так как $(5, 6) = 1$ и  $3 \equiv 3 \mod 5$, $3\equiv 3 \mod 6$, то по Китайской теореме об остатках $k \equiv 3 \mod 5\cdot 4$. Тогда:
		  \begin{equation*} 
		  \begin{cases}
		  k \equiv 1 \mod 6 \\
		  k \equiv 3 \mod 20
		  \end{cases}
		  \end{equation*}
		  Поймем, что $k \ne 3$, $k \ne 23$, и что $k=43$ является минимальным натуральным решением системы (43 - наименьшее после 3 и 23 число дающее в остатке 3 при делении на 20). Тогда минимальное $n = 7 \cdot 43 = 301$.
		  
		  \textbf{Ответ:} 301.
		  
		  
		  \section {Задача 7}
			   Согласно китайской теореме об остатках для любых попарно взаимно-простых $a_1, a_2, ..., a_n$ и для любых $r_1, r_2, ..., r_n$ таких, что $0 \leqslant r_i < a_i$, существует единственный $x$, такой, что $0 \leqslant x < M = \prod_{1}^{n} a_i$ и $x$ является решением системы: 
			\begin{equation*} 
				   \begin{cases}
				   x \equiv r_1 \mod a_1 \\
				   x \equiv r_2 \mod a_2 \\
				   \vdots
				   \\
				   x \equiv r_n \mod a_n
                   \end{cases}
			\end{equation*}
			
			И любой $x^\prime \equiv x \mod \prod_{1}^{n} a_i$ так же является решением этой системы. Причем каждому набору остатков соответствует один $0 \leqslant x < \prod_{1}^{n} a_i$.
			
			В свете этого условие задачи можно сформулировать так: <<Сколько существует $x$, таких что $0 \leqslant x \leqslant 2310000$  и для всех $a_i \in {2,3,5,7,11}$ верно $x \equiv r_{ij} \mod a_i$, где $0 < r_{ij} < a_i$>>
			
			Т.е найти решения следующей системы по модулю $2310000$
			\begin{equation*} 
			\begin{cases}
			x \equiv r_1 \mod 2 \\
			x \equiv r_2 \mod 3 \\
			x \equiv r_3 \mod 5 \\
			x \equiv r_4 \mod 7 \\
			x \equiv r_5 \mod 11
			\end{cases}
			\end{equation*}
			
			($x$ является решением системы $\Longleftrightarrow$ $x$ не делится на 2, 3, 5, 7, 11 и меньше 2310000, т.е. принадлежит множеству тех чисел которые нужно посчитать по оригинальному условию). Но поймем, что для любого набора ${r_1, r_2, ..., r_5}$ существует единственное решение по модулю $2 \cdot 3 \cdot 5 \cdot 7 \cdot 11 = 2310$ (т.к. все решения попарно сравнимы по этому модулю) и соответственно 1000 решений по модулю $2310 \cdot 1000$. Всего наборов остатков не равных 0: $ (2 - 1) \cdot (3 - 1) \cdot (5 - 1) \cdot (7 - 1) \cdot (11 - 1) = 480$, тогда решений системы по модулю $ 2310 \cdot 1000$ ровно $480000$.
		
		 \textbf{Ответ:} 480000.
		 
		 
		  \section {Задача 8}
		  
		  Если число $n$ взаимно-просто с 10, то по теореме Эйлера:
		  $$ 10^{\varphi(n)} \equiv 1 \mod n$$
		  
		  Поймем, что число $E_{\varphi(n)} = \frac{10^{\varphi(n)} - 1}{9}$ - репьюнит, и что $9E_{\varphi(n)} \equiv 0 \mod n$. Рассмотрим репьюнит из $9\varphi(n)$ единиц. 
		  
		  $$ E_{9\varphi(n)} = E_{\varphi(n)} \cdot 10^{8\varphi(n)} + ... + E_{\varphi(n)} \cdot 10^{\varphi(n)} + E_{\varphi(n)} = $$
		  $$=  E_{\varphi(n)} ( 9E_{\varphi(8n)} + 1 + 9E_{\varphi(7n)} + 1 + ... +  9E_{\varphi(n)} + 1 + 1) = $$
		  $$ = 9E_{\varphi(n)} (E_{\varphi(8n)} + E_{\varphi(7n)} + ... + E_{\varphi(n)} + 1)$$
		  
		  Получается $E_{9\varphi(n)} = 9E_{\varphi(n)} \cdot k \equiv 0 \mod n$. Т.е. для любых $n$ взаимно-простых с 10 существует хотя бы один репьюнит кратный $n$. Поймем, что их на самом деле бесконечно много: пусть $n | E_k$, тогда $n | E_k + 10^k \cdot E_k$, т.е. $n | E_{2k}$ по индукции $n | E_{2^mk} \forall m \in N$.
		  
		  \textbf{Ответ:} Да, их будет бесконечно много.
		  
		  \section {Задача 9}
		  Так как для любых $k > 1$ $(10^k, 7) = 1$ (7 не имеет общих простых множителей с $10^k=2^k5^k$), то по малой теореме Ферма:
			  $$ 7^{\varphi(10^k)} \equiv 1 \mod 10^k $$
		  Тогда в частности для $k=4:$
			  $$ 7^{\varphi(10^4)} \equiv 1 \mod 10^4 $$
		  Т.е. $7^{\varphi(10^4)} = 10^4\cdot k + 1$, т.е. оканчивается на 0001. Для пущей уверенности посчитаем $\varphi(10^4) = \varphi(2^4)\varphi(5^4) = 2^3\cdot(2 - 1) 5^3 \cdot(5 - 1) = 4000$.
		   
		  \textbf {Ответ:} Да существует, например $7^4000$.
\end{document}