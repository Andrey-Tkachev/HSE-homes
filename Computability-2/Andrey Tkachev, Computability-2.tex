\documentclass{article}

\usepackage {titlesec}
\usepackage[utf8]{inputenc}
\usepackage[english,russian]{babel}

% Offset setup %
\usepackage[left=15mm,
            top=15mm, 
            right=10mm,
            bottom=15mm, nohead, nofoot]{geometry}

% Maths packages %
\usepackage{amssymb}
\usepackage{amsmath}

% Special symbols %
\usepackage{wasysym}
\usepackage{cancel}
\usepackage{graphicx}
\usepackage{numprint}


% Pretty algorithm %
\usepackage{tikz}
\usepackage{algpseudocode}
\usepackage[linesnumbered,boxed]{algorithm2e}
\titlespacing*{\section}{\parindent}{*4}{*4}
\title{Домашнее задание 21}
\author{Ткачев Андрей, группа 166}
\date{\today}
\graphicspath{{pictures/}}

% Other%
\newcommand{\pr}{^{\prime}}
\newcommand{\ppr}{^{\prime\prime}}
\newcommand{\xp}{x^{\prime}}
\newcommand{\xpp}{x^{\prime\prime}}
\newcommand{\xppp}{x^{\prime\prime\prime}}
% Alias %
\newcommand{\pair}[2]{(#1,\ #2)}
\newcommand{\andi}{$ и $}
\newcommand{\xor}{\oplus}
%\newcommand{\xor}{\oplus}%

% Fracs %
\newcommand{\half}[1]{\frac{#1}{2}}
% Pretty Num letters%
\newcommand{\N}{\mathbb{N}}
\newcommand{\R}{\mathbb{R}}
\newcommand{\Q}{\mathbb{Q}}
\newcommand{\M}{\mathbb{M}}
\newcommand{\conti}{2^{\N}}

% Vector graphics %
\definecolor{white}{rgb}{1.0, 1.0, 1.0}
\definecolor{black}{rgb}{0.0, 0.0, 0.0}
\definecolor{royalblue}{RGB}{255,170,128}
\definecolor{lgr}{RGB}{168, 228, 160}
\definecolor{lblue}{RGB}{24,205,255}

\usetikzlibrary{decorations.markings}
\usetikzlibrary{shapes.geometric}

\pgfdeclarelayer{edgelayer}
\pgfdeclarelayer{nodelayer}
\pgfsetlayers{edgelayer,nodelayer,main}

\tikzstyle{none}=[inner sep=0pt]
\tikzstyle{node}=[circle,fill=white,draw=black,thick]
\begin{document}
    \maketitle
	\paragraph{Задача 1}
		Пусть $A = \{U(p,p) : p\in \N\}$. Покажем, что $\forall a \in \N: a \in A$.
        Раccмотрим константную функцию $g(x) = a$. Данная функция очевидно вычислима, а т.к. $U$ - у.в.ф., то $\exists p\in \N: g(x) = U(p, x)$. Но тогда, $U(p, p) = a$. Таким образом, для любого натурального числа $a$ найдется $p \in \N$, такое, что $U(p, p) = a$. Значит $A = \N$. 
	\paragraph{Задача 2}
		\textbf{(a)} Пусть уаказанное множество $S$ -- разрешимо; Пусть функция $f(x)$ такова 
            \begin{equation*}
             f(x) = 
             \begin{cases} 
               U(k, k^2) + 1 &\text{если $x = k^2, k\in S$}\\
               0 &\text{если $x \ne k^2, \forall k \in \N$} \\
               1 &\text{если $x = k^2, k \notin S, k\in \N$} \\
             \end{cases}
            \end{equation*}

        Тогда $f(x)$ --- вычислима, т.к. вычислима проверка $x = k^2$ и проверка $k \in S$ --- вычислима в силу предположения разрешимости множества $S$:
        Тогда $\exists p: f(x)= U(p, x) \forall x\in \N$. 

        Пусть $p \in S$. Тогда, $U(p, p^2) = f(p^2) = U(p, p^2) + 1$.  Но $U(p, p^2)$ --- определено, при $p \in S$ $\Rightarrow$ $p \notin S$.

        Пусть $p \notin S$. Тогда, $U(p, p^2) = undef = f(p^2) = 1$ ---  противоречие. 

        Значит, S --- неразрешимо.

        \textbf{Ответ:} да, верно.

        \textbf{(б)} Пусть $U$ --- какая-то у.в.ф.; Пусть послдовательность $(\delta_i)$ такова, что $\delta_i$ --- $(i + 1)^{\text{ое}}$ по счету число, не являющееся полным квадратом(т.е. $\delta_0 = 2$, $\delta_1 = 3 ...$). Понятно, что множество $\{\delta_i\}$ --- разрешимо (для любого числа можно проверить, является ли оно полным квадратом). Рассмотрим функцию
            \begin{equation*}
             V(p, x) = 
             \begin{cases} 
               0 &\text{$p \notin (\delta_i)$}\\
               U(i, x) &\text{$p = \delta_i$}
             \end{cases}
            \end{equation*}
        $V(p, x)$ --- у.в.ф., т.к. вычислима(проверки вычислимы и $U$ --- у.в.ф.) и если $f(x)$ --- некоторая вычислимая функция, то $f(x) = U(n, x)$, тогда $V(\delta_n, x) = U(n, x) = f(x)$. Но поймем, что $\forall p \in \N$ $V(p^2, p) = 0$, т.е. определено, значит для данной у.в.ф. множество $\{p | V(p^2, p)$ - определено$\}$ совпадает с $\N$, а значит разрешимо.

        \textbf{Ответ:} нет, не верно.

	\paragraph{Задача 3}
		Пусть $U$ --- некотрая у.в.ф., множество $A$ --- бесконечное и разрешимое (и, как следствие перечислимое). Пусть перечислитель $A$ перечисляет множество в порядке $a_0, a_1, \ldots, a_n, \ldots$. Построим функцию $V$:
            \begin{equation*}
             V(p, x) = 
             \begin{cases} 
               notdef &\text{$p \notin A$}\\
               U(i, x) &\text{$\exists i:p = a_i$}
             \end{cases}
            \end{equation*}
        Т.к. $A$ --- разрешимо, то $V$ --- вычислима. Покажем, что $V$ --- у.в.ф.: если $f(x)$ --- вычислимая функция, то $\exists p \forall x: f(x) = U(p, x) = V(a_p, x)$ ($a_p$ существует в силу бесконечности $A$), т.е. существует $k$, что $\forall x: f(x) = V(k, x)$. Рассмотрим функцию $g(x) = V(x, x^2)$. Данная функция вычислима, т.к. $V$ --- у.в.ф., а значит ее область определения --- перечислимое множество (одно из определений перечислимого множества). Но как мы помним по задаче $2(a)$ область определения такой функции --- неразрешима. Осталось заметить, что область определения есть подмножество $dom(V) \subset A$ (по построению $V$ --- определена только на каком-то подмножестве $A$, причем $dom(V) \ne A$ т.к. $dom(U) \ne \N$). Значит мы построили неразрешимое перечислимое подмножество в $A$.

	\paragraph{Задача 4}
		Обозначим за $S$ бесконечное подмножество $\N$.

        Докажем прямое следствие: $S$ -- бесконечно разрешимо $\Rightarrow$ $\exists f(x): \N \rightarrow \N$ --- всюду опрделенная возрастающая вычислимая функция, $E(f) = S$. Построим такую функцию, но для начала оговорим, что если множество $S$ разрешимо, то и перечислимо в порядке возрастания и без повторений (перечислние путем простой иттерацией по натуральным и вычслимой проверкой на вхождение в $S$) $s_0 < s_1 < s_2 < \ldots$: $f(x) = s_x$.
        Данная функция всюдуопределена, т.к. $S$ --- бесконечно, и очевидно возрастает, по природе последовательности $(s_i)$ и область значения совпадает с $S$ (т.к. любое $s\in S$  входит в $(s_i)$).

        Докажем обратное следствие: $f(x): \N \rightarrow \N$ --- всюду опрделенная возрастающая вычислимая функция с областью значений $S$ $\Rightarrow$ S -- бесконечное разрешимое множество.

        Бесконечность множества $S$ следует из тотальности и возрастания $f(x)$. Покажем, что $S$ --- разрежимо. Пусть нам необходимо узнать, принадлежит ли $a\in \N$ множеству $S$. За конечное число шагов найдем минимальное $x_m$, такое, что $f(x_m) > b$ (функция возрастает, значит существует аргумент начиная с которого функция привысит любую наперед заданную константу). Тогда $a \in S$ $\Leftrightarrow$ $f(x_m - 1)$ и $x_m - 1 \ge 0$ (действительно, если $f(x_m-1) = a$, то $f(x_m) > a$ --- минимальное число строго большее $a$ и $f(0) \le a$). Значит $\forall x \in N$ за конечное число шагов можно определить принадлежность $x$ мн-ву значений $S$.

        Значит $f(x): \N \rightarrow \N$ --- всюду опрделенная возрастающая вычислимая функция с областью значений $S$ $\Leftrightarrow$ S -- бесконечное разрешимое множество.

	\paragraph{Задача 5}
		Пусть $A$ -- бесконечное перечислимое множество. Тогда существует перечислитель, который перечисляет элементы $A$ без повторения (существует какой-то перечислитель, заставим его запоминать уже выведенные числа, и не будем перечислять уже выведенное). Пусть перечислитель перечисляет эл-ты в последовательности $a_0, a_1, a_2\ldots$. Зададим последовательность $(s_i)$ следующим образом:

        \begin{equation*}
          s_i = 
          \begin{cases} 
            a_0 &\text{$i = 0$}\\
            a_j &\text{$s_{i - 1} = a_k$, тогда $a_j$ --- первое число в $(a_i)$, что $j > k \andi a_j > a_k$}
          \end{cases}
        \end{equation*} 

        Поймем, что последовательность $(s_i)$ бесконечна и возрастает, т.к. для любого $a_i \in A$ существует $j > i: a_j > a_i$ (иначе существует бесконечно много натуральных числел меньших $a_i$, что не так). Но тогда рассмотрим функцию $f(x) = s_x$. Она возрастает и ее область значений --- множество $\{s_i\}$, а значит, как следует из задачи 5, множество $\{s_i\}$ -- разрешимо. Т.е. мы получили бесконечное разрешимое подмножество в $A$.
	
    \paragraph{Задача 6}
    Для каждой у.в.ф. $U(p, x)$ существует вычислимая отладочная функция $F(p, n, t)$, такая, что $F$ -- тотальна и не убывает по $t$ и $F(p, x, t) = 0 \Leftrightarrow $ алгоритм, вычисляющий $U(p, x)$ не остановится за $t$ шагов. Мы знаем, что множество $\N^3$ --- перечислимо. Тогда будем перечислять всевозможные тройки $(p, x, t)$ и выводить $p$, тогда когда на входе $(p, x, t)$ вычислимая всюдуопределенная $F$ принимает значение 1. Тогда мы перечислим ровно те $p$, при которых $U(p, x)$ определена хотябы на одном $x$ (т.к. если $p$ перечислено, то перечислена тройка $(p,x,t): F(p,x,t) = 1$, т.е. в точке $x$ $U_p$ определено; Обратно: если $U_p$ определено на входе $x$, то алгоритм вычисляющий $U(p, x)$ остановится за конечное число шагов $k$, а значит $F(p, x, t)=1$, но тройка $(p, x, t)$ будет перечислена перечислителем $\N^3$, заначит число $p$) будет выведено).
\end{document}