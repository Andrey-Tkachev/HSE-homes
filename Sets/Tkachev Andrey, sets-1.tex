\documentclass{article}
\usepackage{cancel}
\usepackage[utf8]{inputenc}
\usepackage {titlesec}
\usepackage[english,russian]{babel}
\usepackage{amssymb}
\usepackage{amsmath}
\usepackage{graphicx}
\usepackage{numprint}
\graphicspath{{pictures/}}

\titlespacing*{\section}{\parindent}{*4}{*4}

\title{Домашнее задание 15}
\author{Ткачев Андрей, группа 166}
\date{\today}
\newcommand{\niton}{\not\owns}
\newcommand{\pr}{^{\prime}}
\newcommand{\ppr}{^{\prime\prime}}
\newcommand{\xp}{x^{\prime}}
\newcommand{\xpp}{x^{\prime\prime}}
\newcommand{\xppp}{x^{\prime\prime\prime}}
\newcommand{\pair}[2]{(#1,\ #2)}
\newcommand{\andi}{$ и $}
\newcommand{\half}[1]{\frac{#1}{2}}
\newcommand{\eqp}{\sim}
\newcommand{\N}{\mathbb{N}}
\newcommand{\M}{\mathbb{M}}
\newcommand{\T}{\mathbb{T}}
\newcommand{\Y}{\mathbb{Y}}

\begin{document}
    \maketitle
    \paragraph{Задача 1.}
    $A \setminus B$ - бесконечно $\Rightarrow \exists C:\ C \subseteq A \setminus B \subseteq A$ - счетно. Тогда $C \cap B = \emptyset$. Пусть $C = \{c_0, c_1, \ldots\}$, $B = \{b_0, b_1, \ldots\}$. Построим биекцию $f(x): A \setminus B \rightarrow  A$.

    \begin{equation*}
        f(x) = 
         \begin{cases}
           x &\text{$x \notin C$}\\
           b_{k} &\text{$x = c_{2k}$}\\
           c_{k} &\text{$x = c_{2k + 1}$}
         \end{cases}
    \end{equation*} 

    Поймем, что $f(x)$ действительно биекция. 

    Очевидно, что $f(x)$ - сюръекция (любой элемент $a \in A$ либо пренадлежит $B$, либо $C$, либо не принадлежит им обоим, значит по $a \in A$ можно востановить $a\pr \in A \setminus B$)

    Если $f(\xp) = f(\xpp)$, то либо $\xp, \xpp \notin C$, либо $\xp = c_{2k},\ \xpp = c_{2m}$, 
    либо $\xp =c_{2k + 1},\ \xpp = c_{2m + 1}$, так как множества $B$, $C$ и 
    $A \setminus (B \cup C)$ попарно не пересекаются, а $f(x)$ отображает $x$ в одно из них. 
    Если $\xp, \xpp \notin C$, то $f(\xp) = f(\xpp) = \xp = \xpp$. 
    В противном случае $f(\xp) = c_{2k} = c_{2m} = f(\xpp)$, такое возможно только когда $k = m$. Аналогично $f(\xp) = f(\xpp) \Rightarrow \xp=\xpp$. Т.е. отображение не склеивающее.

    Таким образом между $A$ и $A \setminus B$ сущ. биекция, значит они равномощны.

    \textbf{Ответ:} Верно.

    \paragraph{Задача 2.}
    Расмотрим такой пример: $A = \mathbb{N}, B = \mathbb{N} \setminus \{0\}$. Понятно, что $A \sim \N \sim B$ (между $B$ и $\N$ можно установить естественную биекцию $n \rightarrow n - 1$, поэтому $B$ - счетно; $A$ - бесконечно, т.к. $\N$ - бесконечно). 

    Но тогда $A \triangle B = \N \triangle \N \setminus \{0\} = \{0\} \nsim A$.
    
    \textbf{Ответ:} Неверно.

    \paragraph{Задача 3.}
    Заметим, что $A \setminus B$ - бесконечно 
    $\Rightarrow \exists C:\ C \subseteq A \setminus B \subseteq A$ и $C$ - счетно. Тогда 
    $C \cap B = \emptyset$. Пусть $C = \{c_0, c_1, \ldots\}$, $B = \{b_0, b_1, \ldots\, b_{n - 1}\}$ ($|B| = n$). Построим биекцию $f(x): A \setminus B \rightarrow  A$.

    \begin{equation*}
        f(x) = 
         \begin{cases}
           x &\text{$x \notin C$}\\
           b_{k} &\text{$x = c_{k},\ k < n $}\\
           c_{k} &\text{$x = c_{k - n}, k \ge n$}
         \end{cases}
    \end{equation*}

    Нетрудно убедится, что $f(x)$ всюду определенная функция, инъекция и сюръекция 
    (всюду определенность очевидна; 
    инъективность достигается за счет того, что если $f(x) = f(x\pr)$, то либо $x, x\pr \in C \Rightarrow x = x\pr$, либо $x = c_{a}, x = c_{b}$ причем $a \andi b$ либо больше $n$ либо нет, но в силу построения $a = b \Rightarrow x = x\pr$;
    сюръективность следует из того, что любой элемент в $A$ либо $\in C$, либо $\in B$, либо $\in A \setminus (C \cup B)$ и для каждой из опций возможно востановить соответсвующий член из $A \setminus B$), 
    а значит и биекция. Тогда $A \setminus B \sim A$.

    \textbf{Ответ:} Верно.

    \paragraph{Задача 4.}
    Пусть множество интервалов $S$. Каждый интервал определяется парой различных вещественных чисел. 
    Между любыми двумя вещественными числами можно найти рациональное (как мы помним из курса математического анализа). Т.е. любому интервалу можно однозначно поставить в соответствие какое-либо рациональное число, ему принадлежащее (например среденее арифмитическое рационального приближения левой границы с избытком и правой с недостатком), причем для двух непресекающихся интервалов эти числа обязательно будут разными.

    Тогда существует инъекция из $S$ в $\mathbb{Q}$. Т.е. существует биекция между $S$ и проекций $S$ в $\mathbb{Q}$, т.е. $S$ равномощно какому-то подмножеству $\mathbb{Q}$. $\mathbb{Q}$ - счетно, значит $S$ - не более чем счетно, значит конечно или счетно.

    \paragraph{Задача 5.}
    Всякое бесконечное множество $A$ содержит счетное множество $B$. Докажем теперь, что всякое счетное множество $B$ содержит в себе бесконечное число счетных множеств.

    Раз $B$ счетно, то бесконечно. Пусть $B = \{b_0, b_1, \ldots\}$. Пусть множество $\mathbb{P} = \{p_0 = 2, p_1 = 3, p_2 = 5 \ldots\}$ - множество простых чисел, которое, как мы знаем, счетно. Рассмотрим множество 
    $C = \{\{x_i = b_{p_0^i}\}, \{x_i = b_{p_1^i}\}, \ldots, \{x_i = b_{p_2^i}\} \ldots \}$, где $i \in \mathbb{N} \setminus \{0\}$. 

    Поймем, что  если $c_1, c_2 \in C$, то либо $c_1 = c_2$, либо $c_1 \cap c_2 = \emptyset$ (т.к. каждый элемент из $C$ - множество элементов из $B$ с индексами предствимыми в виде $p_k^n$, т.е. наличие общих элементов в $c_1$ и $c_2$ влечет равенство оснований по которому выбирали 
    (ведь основания простые числа и $p_k^n = p_t^l \Rightarrow p_k = p_t$ и $n = l$ по ОТА)).
    Но $C$ - счетно, т.к. $C \sim \mathbb{P} \sim \mathbb{N}$, значит бесконечно.
    При этом $c \in C$ - бесконечное подмножество $B$ ($c = \{b_{p_k^1}, b_{p_k^2}, b_{p_k^3} \ldots\}$, т.е. $c \sim \mathbb{N}$ так как существует естественная биекция: степень $\rightarrow$ натуральное число), которое счетно, значит $c$ - счетно.

    \paragraph{Задача 6.}
    Поймем, что чтобы описать периодическую функцию с перидом $T$ достаточно знать, 
    какие значения она принимает в целых точках на отрезке $[0, T - 1]$. 

    Пусть $S_T$ - множество всех периодических функций $f: \mathbb{Z} \rightarrow \mathbb{Z}$ с периодом $T$ ($T \in N \setminus \{0\}$). 
    Тогда поймем, что существует биекция $S_T \rightarrow \mathbb{Z}^T$
    (фунции с периодом $T$ взаимно однозначно соответсвует вектор из $\mathbb{Z}^T$, координаты вектора - упорядоченные значения принимаемыми функцией из $S_T$ на $[0, T - 1]$).
    Но тогда $S_T \sim \mathbb{Z}^T \sim \mathbb{Z} \sim \mathbb{N}$, а значит $S_T$ - счетно.

    Поймем, что множество $S$ всех периодических функций $\mathbb{Z} \rightarrow \mathbb{Z}$ можно записать как $S = \cup_{T = 1}^{\infty} S_T$. Тогда $S$ - счетное объединение счетных множеств, а значит счетно.

    \paragraph{Задача 7.}
    Пусть $\N^*$ - множество всех конечных последовательностей натуральных чисел, 
    а     $\M^*$ - множество всех конечных последовательностей натуральных не содержащих нулей. 
    
    Построим биекцию $f: \N^* \rightarrow \M^*$.
    Пустой последовательности из $\N^*$ сопоставим ее же в $\M^*$.
    Каждой не пустой последовательности $S = (s_0, s_1, \ldots, s_n) \in \N^*$ взаимно однозначно сопоставим последовательность

    $$S\pr = (s_0 + 1, s_1 + 1, \ldots, s_n + 1) \in \M^*.$$

    Поймем, что указанным образом мы дествительно установим взаимно однозначное соответсвие между последовательностаями в $\N^*$ и в $\M^*$ 
    (действительно, такое соответствие 
    всюду определено и функционально
    (по любой последовательности $S$ можно однозначно построить $S\pr$), 
    инъективно 
    (если $a \rightarrow S\pr$ и $b \rightarrow S\pr$, то $a_i = S_i - 1 = b_i,\ $где $i$ - индекс элемента последовательности)),
    сюръективно(по любой последовательности из $\M^*$ можно сказать из какой последовательности она была получена, просто вычтя единицу их каждого члена)).

    Таким образом

    \begin{equation*}
        f(x) = 
         \begin{cases}
           x &\text{$x$ - пустая последовательность}\\
           (x_0 + 1, \ldots, x_n + 1) &\text{$x = \{x_0, x_1, \ldots, x_n\}$}\\
         \end{cases}
    \end{equation*}


    Пусть теперь $\T$ - множество всех возрастающих последовательностей натуральных чсел, не содрежащих 0.
    Построим биекцию $g: \M^* \rightarrow \T$. Как и прежде пустой последовательности сопоставим пустую последовательность.
    Если же $S = \{s_0, \ldots, s_n\} \in \M^*$ непустая последовательность, то сопоставим ей следущую последовательность 

    $$S\pr = (s_0, s_0 + s_1, s_0 + s_1 + s_2, \ldots, \sum_{i=0}^{n} s_i)$$ 
    
    Т.е. каждый член $S$ задает в новой последовательности <<отступ>> от предыдущего (например последовательности $(1, 3, 2, 1)$ будет сопоставлена последовательность $(1, 4, 6, 7)$).

    Очевидно, что $S\pr \in \T$, ведь эта последоваетльность возрастает по построению, т.к. каждый член отличен от предыдущего на положительную ненулевую величину. То есть $g$ имеет вид

    \begin{equation*}
        g(x) = 
         \begin{cases}
           x &\text{$x$ - пустая последовательность}\\
           (x_0, x_0 + x_1, \ldots, \sum_{i=0}^{n} x_i) &\text{$x = (x_0, x_1, \ldots, x_n)$}\\
         \end{cases}
    \end{equation*}

    Поймем, что $g$ всюдуопределено и функционально (указанными операциями мы для любой последовательности из $\M^*$ однозначно определяем последовательность из $\T$).

    Почему $g$ - инъективно? Пусть $g\left((a_0, \ldots, a_n)\right) = g\left((b_0, \ldots, b_m)\right) = (c_0, \ldots, c_k)$. Очевидно, что $k = n = m$, т.к. последовательности длины $n + 1$ сопостаявляется последовательность длины $n + 1$. Также $c_0 = a_0 = b_0$. Но тогда $c_1 = a_0 + a_1 = b_0 + b_1 \Rightarrow b_1 = b_2$. Применяя метод индукции получим, что $\forall i <= n: a_i = b_i \Rightarrow (a_0, \ldots, a_n) = (b_0, \ldots, b_m)$.

    Перейдем к сюръекции. Пусть $S = (s_0, \ldots, s_n) \in \T$. Поймем, что $S\pr = (s_0, s_1 - s_0, s_2 - s_1, \ldots, s_n - s_{n - 1}) \in \M^*$ (последовательность $S$ строго возрастает, значит $S\pr$ состоит из положительных чисел $\ge 1$). Но $g(S\pr) = S$. Значит $g$ - сюръекция.

    Таким образом $g$ - биекция.

    Заключительный штрих. Построим биекцию $h: \T \rightarrow \Y$, где $\Y$ - множество всех строговозрастающих конечных последовательностей натуральных чисел. Постоение тривиально и практически повторяет построение $f$:

    \begin{equation*}
        h(x) = 
         \begin{cases}
           x &\text{$x$ - пустая последовательность}\\
           (x_0 - 1, x_1 - 1, \ldots, x_n - 1) &\text{$x = (x_0, x_1, \ldots, x_n)$}\\
         \end{cases}
    \end{equation*}

    Всюдуопредленность, функциональность, инъективность и сюръективность $h$ очевидны.

    При этом рузультат применения $h$ - строговозрастающая последовательность, которая возможно содержит нулевые элементы (вернее один нулевой элемент - первый). 

    Тогда $h \circ g \circ f: \N^* \rightarrow \Y$ - требуемая биекция (композиция биекций - биекция). Более формально:

    \begin{equation*}
        h \circ g \circ f (x) = 
         \begin{cases}
           x &\text{$x$ - п. п.}\\
           ((x_0 + 1) - 1, ((x_0 + 1) + (x_1 + 1)) - 1, \ldots,\\ \sum_{i=0}^{n} (x_i + 1) - 1) &\text{$x = (x_0, \ldots, x_n)$}\\
         \end{cases}
    \end{equation*}

    Или в упрощенном виде:

    \begin{equation*}
        h \circ g \circ f (x) = 
         \begin{cases}
           x &\text{$x$ - п. п.}\\
           (x_0 - 1,  \ldots, \sum_{i=0}^{k} (x_i) + k - 1, \ldots,\\ \sum_{i=0}^{n} x_i + n - 1) &\text{$x = (x_0, x_1, \ldots, x_n)$}\\
         \end{cases}
    \end{equation*}

    \textit{Примечание.} Во всех определениях функция запись $f(x), g(x), etc$ подразумевает, что $x$ находится в области определения.
\end{document}