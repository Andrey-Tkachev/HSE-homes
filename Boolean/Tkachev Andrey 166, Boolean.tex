\documentclass{article}

\usepackage {titlesec}
\usepackage[utf8]{inputenc}
\usepackage[english,russian]{babel}

% Offset setup %
\usepackage[left=15mm,
            top=15mm, 
            right=10mm,
            bottom=15mm, nohead, nofoot]{geometry}

% Maths packages %
\usepackage{amssymb}
\usepackage{amsmath}

% Special symbols %
\usepackage{wasysym}
\usepackage{cancel}
\usepackage{graphicx}
\usepackage{numprint}
\graphicspath{{pictures/}}

\titlespacing*{\section}{\parindent}{*4}{*4}



\title{Домашнее задание 17}
\author{Ткачев Андрей, группа 166}
\date{\today}

% Other%
\newcommand{\pr}{^{\prime}}
\newcommand{\ppr}{^{\prime\prime}}
\newcommand{\xp}{x^{\prime}}
\newcommand{\xpp}{x^{\prime\prime}}
\newcommand{\xppp}{x^{\prime\prime\prime}}
% Alias %
\newcommand{\pair}[2]{(#1,\ #2)}
\newcommand{\andi}{$ и $}
\newcommand{\xor}{\oplus}
%\newcommand{\xor}{\oplus}%

% Fracs %
\newcommand{\half}[1]{\frac{#1}{2}}
% Pretty Num letters%
\newcommand{\N}{\mathbb{N}}
\newcommand{\R}{\mathbb{R}}
\newcommand{\Q}{\mathbb{Q}}
\newcommand{\M}{\mathbb{M}}
\newcommand{\conti}{2^{\N}}

\begin{document}
    \maketitle
    \paragraph{Задача 1.}
    Пусть множество $\M$ - множество бесконечных последовательностей $0,\ 1,\ 2$, в котрых ни один символ не идет два раза подряд.

    Покажем, что $\conti \apprle \M$. Каждому символу в бесконечной двоичной последовательности из $\conti$ сопоставит последовательность из $\M$ по следующему правилу кодирования:

    \begin{itemize}
        \item $0 \rightarrow 01$
        \item $1 \rightarrow 02$
    \end{itemize}

    Так, например, из последовательности $0110\ldots$ будет получена последовательность $01020201\ldots$. 
    Понятно, что это отображение инъективно (если две двоичные последовательности различаются в $i$-ом бите, то конечные последовательности будут различаться в $2i + 1$-ом символе). 
    При этом, очевидно, что в конечной последовательности никакие две цифры не идут два раза подряд. 
    Таким образом существует инъекция $\conti \rightarrow \M$.

    Теперь докажем, что $\M \apprle \conti$. Каждому символу в бесконечной последовательности из $\M$ сопоставит последовательность из $\M$ по следующему правилу кодирования:

    \begin{itemize}
        \item $0 \rightarrow 00$
        \item $1 \rightarrow 01$
        \item $2 \rightarrow 11$        
    \end{itemize}

     Так, например, из последовательности $012\ldots$ будет получена последовательность $000111\ldots$. Инъективность такого отображения проверяется ровно так же, как уже было показано выше. 

     Таким образом, по теореме Кантора-Берштейна, $\M \sim \conti$.

     \paragraph{Задача 2.}
     Пусть множество отношений эквивалентности на натуральных числах - $\M$.

     Поймем, что отношение эквивалентности разбивает натуральный ряд на не более чем счетное число классов эквивалентности (иначе из каждого класса можно выбрать по элементу и получить более чем счетное подмножество натурального ряда, что невозможно).
     Тогда классы эквивалентности можно пронумеровать числами натурального ряда(например, в порядке возрастания наименьших членов).

     Значит, конкретному отношению эквиалентности можно сопоставить однозначно функцию $\N rightarrow \N$ (такое отображение будет инъективно; для любых двух неизоморфных разбиений на классы эквивалентности посмотрим на первый такой элемент $i$, принадлежит разным классам эквивалентности; он принадлежит разным классу $k \le i$ в одном отношении, в другом отношении он принадлежит классу $k\pr \ne k$, значит $i$-ый символы в конечных последовательностях различны). Таким образом $\M \apprle \N^{\N} \sim \conti$.

     Покажем теперь, что $\conti \apprle \M$. Построим по двоичной последовательности $S \in \conti$ разбиение на классы эквивалентности:

    \begin{itemize}
        \item $S_i = 0 \Rightarrow $ число $i + 1$ не эквивалентно 0.
        \item $S_i = 1 \Rightarrow $ число $i + 1$    эквивалентно 0.    
    \end{itemize}

    Т.е. двоичной последовательностью мы разбиваем множество натуральных на два класса эквивалентности --- содержащий и не содержащий 0. Каждое такое разбиение однозначно определяет последоваетльность, а последовательность однозначно задает разбиение $\Rightarrow$ существует инъекция $\conti \rightarrow \M$.

    Тогда, по теореме Кантора-Берштейна, $\conti \sim \M$.

     \paragraph{Задача 3.}
     Пусть $p(A) = 2^{A}$ - множество всех подможетв $A$.
     Пусть множество отношений эквивалентности на множестве действительных чисел - $\M$.

     Покажем, что $p(\R) \apprle \M$. Заметим, что $p(\R) \sim p(\R \setminus \{0\})$. 
     Тогда, возьмем подмножество из $\R \setminus \{0\}$. Объявим, что все числа в него входящие --- эквивалентны $0$, а все числа, входящие в его дополнение до $\R \setminus \{0\}$ --- не эквивалентны $0$.
     Тогда для двух различных подмножеств в $\R \setminus \{0\}$ мы получим два различных разбиения на классы эквивалентности. Тогда $p(\R) \sim p(\R \setminus \{0\}) \apprle \M$.

     Докажем, что и $\M \apprle p(\R)$. Рассмотрим множество $S$ классов эквивалентности для отношения из $\M$. Каждому классу эквивалентности из $S$ принадлежит хотябы один элемент из $\R$. Тогда для каждому классу в $S$ можно сопоставить элемент из $\R$, ему принадлежащий. Для каждого класса в $S$ выберем этот элемент и составим из них множество $P \apprle \conti$ (подмножество континуума). Тогда это отношение эквивалентности определяется функцией $\R \rightarrow P$ (каждому числу в $\R$ сопоставляем число выбранное для его класса эквивалентности). Тогда кажодое отношение эквивалентности можно перевести в функцию $\R \rightarrow \R$. Тогда получаем, что $\M \apprle \R^{\R} \sim (\conti)^{\R} \sim 2^{\N \times \R} \sim 2^{\R} = p(\R)$.

     Значит, по т. Кантора-Берштейна $\M \sim p(\R)$.

     \paragraph{Задача 4.}
     Посмотрим, при каких значениях аргументов функция принимает значение 1.
    
     \begin{equation*}
       f(\vec{x}) = 1 \Leftrightarrow
         \begin{cases}
           (x_1 \vee x_2) = 1, \\
           (\bar{x}_1 \vee x_3) = 1, \\
           (\bar{x}_2 \vee x_5) = 1, \\
           \ldots \\
           (\bar{x}_7 \vee x_9) = 1;
         \end{cases}
    \end{equation*}

    В свою очередь $x_1 \vee x_2 = 1 \Rightarrow x_1 \ne 0 \andi x_2 \ne 0$. При этом из равенства $x_1 = 0$ следует, что $x_3 = 1$, из чего следует $x_5 = 1 \ldots x_9 = 1$.

    Аналогично, если $x_2 = 1$, то $x_2 = x_4 = \ldots = x_8 = 1$.   

    Поймем теперь, что если $x_{2k + 1} = 1$, все последующие $x_{2m + 1} = 1$ и что если
                                $x_{2k} = 1$, все последующие $x_{2m} = 1$.

    Отсюда получаем, что должно быть истинным одно из следующих выражений:

    $$ x_1x_3x_5 x_7 x_9 \cdot x_2x_4x_6x_8$$
    $$ \bar{x}_1x_3x_5 x_7 x_9 \cdot x_2x_4x_6x_8$$
    $$ \bar{x}_1\bar{x}_3x_5 x_7 x_9 \cdot x_2x_4x_6x_8$$
    $$ \bar{x}_1\bar{x}_3\bar{x}_5x_7x_9 \cdot x_2x_4x_6x_8$$
    $$ \bar{x}_1\bar{x}_3\bar{x}_5\bar{x}_7x_9 \cdot x_2x_4x_6x_8$$
    $$ \bar{x}_1\bar{x}_3\bar{x}_5\bar{x}_7\bar{x}_9 \cdot x_2x_4x_6x_8$$
    $$ x_1x_3x_5 x_7 x_9 \cdot \bar{x}_2x_4x_6x_8$$
    $$ x_1x_3x_5 x_7 x_9 \cdot \bar{x}_2\bar{x}_4x_6x_8$$
    $$ x_1x_3x_5 x_7 x_9 \cdot \bar{x}_2\bar{x}_4\bar{x}_6x_8$$
    $$ x_1x_3x_5 x_7 x_9 \cdot \bar{x}_2\bar{x}_4\bar{x}_6\bar{x}_8$$
    
    Тогда запись ДНФ выглядит так:
    $$ \vee_{i=0}^{4} (x_2x_4\ldots x_8 \cdot \bar{x}_1 \ldots \bar{x}_{2i + 1} \ldots x_9)\  \vee_{i=1}^{4} (x_1x_3\ldots x_9 \cdot \bar{x}_2 \ldots \bar{x}_{2i} \ldots x_8)\ \vee (x_1x_2\ldots x_9)$$
     \paragraph{Задача 5.}
     Выразим с помощью штриха Шеффера каждую связку из системы $(\neg,\ \wedge)$, которая является полной.

     $\neg x = x | x$

     Действительно, если $x = 1$, то $\neg (x \wedge x) = 0$, и если $x = 0$, то $\neg (x\wedge x) =1$.

     $x \wedge y = \neg(\neg(x \wedge y)) = \neg(x | y) = (x | y) | (x | y)$

     Таким образом, мы выразили все связки полной системы штрихом Шеффера $\Rightarrow$ он образует полную систему.
     
     \paragraph{Задача 6.}
     Покажем, что любую функцию алгебры логики $f(\vec{x})$ можно представить в КНФ.
     Пусть $f(\vec{x})$ принимает значение истины только на наборах $a_0, \ldots, a_n$. Тогда рассмотрим функцию $g(\vec{x  })$, которая принимает значение истины на всех наборах аргументов, кроме $a_0, \ldots, a_n$. Т.е. $g(\vec{x}) = \bar{f}(\vec{x})$. Для функции $g(\vec{x})$ существует некоторе представление в ДНФ:
     
          $$ g(\vec{x}) = p_0 \vee p_1 \vee \ldots \vee p_k $$

    Где $p_i$ --- конъюнкция некоторых аргументов. Тогда отрицание $g$ можно записать как:

          $$ \bar{g}(\vec{x}) = \overline{p_0 \vee p_1 \vee \ldots \vee p_k} $$

    Применяя $k - 1$ раз закон Де Моргана ($\overline{a \vee b} = \bar{a} \wedge \bar{b} $) получаем.

        $$ \bar{g}(\vec{x}) = \bar{p}_0 \wedge \bar{p}_1 \wedge \ldots \wedge \bar{p}_k $$

    Помня о внутреннем устройстве $p_i$ --- конъюнкция нескольких литералов и законе Де Моргана об отрицании произведения ($\overline{a \wedge b} = \bar{a} \vee \bar{b} $) имеем: $\bar{p}_i$ --- дизъюнкция некторых литералов.
    Но тогда $\bar{g}(\vec{x}) = f(\vec{x})$ есть конъюнкция дизъюнкций литералов, т.е. у $f$ есть представление в КНФ.

     \paragraph{Задача 7.}
     Поймем, что $x_1 \vee x_2 \vee \ldots \vee x_n = x_1 \xor x_2 \xor \ldots \xor x_n \xor x_1x_2 \xor x_1x_3 \ldots \xor x_1x_2\ldots x_n$. Действительно, левая часть выражения принимает значение 0, только если $\vec{x} = \vec{0}$.
     Докажем, что и правая часть ведет себя также.

     Очевидно, что $0 \xor 0 \ldots \xor 0 = 0$. Пусть тогда $x_{i_1}=  \ldots = x_{i_k} = 1$, $k \ge 1$. Тогда ненулевыми будут только те коэффициэнты, что являются конъюнкцией чисел из ${x_{i_j}}$. Таких найдется ровно ${k \choose 1} + {k \choose 2} + \ldots + {k \choose k} = 2^k - 1$. Тогда слагаемых, равных <<единице>> нечетное число, а значит значение выражения равно 1, если хотябы один из аргументов равен 1.

     Количество ненулевых членов в данном полиноме равно ${n \choose 1} + {n \choose 2} + \ldots + {n \choose n} = 2^n - 1$.
     
     \paragraph{Задача 8.}
     Поймем, что функция $MAJ(x_1, x_2, x_3)$ является самодвойственной (действительно, если в наборе $x_1, x_2, x_3$  больше <<единиц>>, то в наборе $\bar{x}_1, \bar{x}_2, \bar{x}_3$ больше 0 и $MAJ(x_1, x_2, x_3) = \neg(MAJ(\bar{x}_1, \bar{x}_2, \bar{x}_3))$, аналогично если среди аргументов больше <<нулей>>).

     Но функция $\neg$ тоже является самодвойственной: $\neg(x) = \neg(\neg(\neg(x)))$.

     Но тогда композиция этих функций тоже самодвойственна, а значит не способна вычислить не самодвойственную функцию, а значит система $(\neg, MAJ(x_1, x_2, x_3))$ не полна. 
\end{document}