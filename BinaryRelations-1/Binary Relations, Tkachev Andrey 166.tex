\documentclass{article}
\usepackage{cancel}
\usepackage[utf8]{inputenc}
\usepackage {titlesec}
\usepackage[english,russian]{babel}
\usepackage{amssymb}
\usepackage{amsmath}
\usepackage{graphicx}
\graphicspath{{pictures/}}

\titlespacing*{\section}{\parindent}{*4}{*4}

\title{Домашнее задание 9}
\author{Ткачев Андрей, группа 166}
\date{\today}

\begin{document}
	\maketitle
	\section {Задача 1}
		Для $\forall a \in A$, $ \forall b \in B$: 
		\begin{itemize}
			\item $\bar{P} = (A \times B) \setminus P$
			\item $P^{-1} = \{(b, a)| (a, b) \in P\}$ 
		\end{itemize}
	Тогда $a \bar{P} b \Rightarrow (a, b) \notin P \Rightarrow (b, a) \notin P^{-1} \Rightarrow \neg a P^{-1} b$. Тогда равенство $P^{-1} = \bar{P}$ невозможно ни для каких бинарных отношений.
	
	\section{Задача 2}
	$P_1$ и $P_2$ транзитивны.
	\subsection{а) Транзитивно ли $\bar{P_1}$?}
	
	Рассмотрим отношение делимости $P_1 = |$ на множестве натуральных чисел. Это отношение транзитивно: если $a | b$ и $b | c$, то $a | c$ ($c = bm = (ak)m$). Но при этом если $a \nmid b$ и $b \nmid c$, то это не означает, что $a \nmid c$ (Пример $a = 3$, $b = 5$, $c = 9$). Т.е. $\bar{P_1} =$ $\nmid$ не транзитивно.
	
	\textbf{Ответ:} не обязательно транзитивно. 
	
	\subsection{б) Транзитивно ли $P_1 \cap P_2?$}
	
	Рассмотрим $M = {(x, y) | (x, y) \in P_1 \cap P_2}$. Если $a M b$ и $b M c$, то  $(a, b) \in M$ и $(b, c) \in M$, а так как $M \subseteq P_1$, то $(a, b) \in P_1$ и $(b, c) \in P_1$ и так как  $M \subseteq P_2$, то $(a, b) \in P_2$ и $(b, c) \in P_2$ $\Rightarrow$ $(a, c) \in P_1$ в силу транзитивности $P_1$  и $(a, c) \in P_2$ в силу транзитивности $P_2$, но тогда $(a, c) \in P_1 \cap P_2 = M$. Т.е. $M$ - транзитивно.
	
	\textbf{Ответ:} да, транзитивно.
	
	\subsection{в) Транзитивно ли $P_1 \cup P_2?$}
	
	Пусть $P_1 = <$, а $P_2 = >$ на множестве действительных (Отношение строго больше(меньше) - транзитивно). Поймем, что если $(a, b) \in P_1 \subset P_1 \cup P_2$ и $(b, c) \in P_2 \subset P_1 \cup P_2$, то не обязательно, что $(a, c) \in P_1 \cup P_2$. Например, если $a = 1$, $b = 3$, $c = 1$ то $(a, c) \notin P_1 \cup P_2$, т.к. $a = c$ - отношение равенства, не пересекается с отношениями строго больше(меньше).  
	
	\textbf{Ответ:} не обязательно транзитивно.
	
	\subsection{в) Транзитивна ли композиция $P_1 \circ P_2?$}
	
	Рассмотрим множество $A = {a, b, c, d, e}$. Определим отношения $P_1$ и $P_2$, так, что $a P_2 d$, $d P_1 b$, $b P_2 e$ и $e P_1 c$. Тогда $P_1$ и $P_2$ - транзитивны (Действительно $\forall i < 2$, $\not {\exists} x, y, z \in A: x P_i y$, $y P_i z \not{\Rightarrow} x P_i z$). 
	
	Также, по определению: $a$ $P_1 \circ P_2$ $b$ и $b$ $P_1 \circ P_2$ $c$. Но при этом не верно, что $a$ $P_1 \circ P_2$ $с$, так как $\not{\exists} x: a P_2 x$ и $x P_1 c$. Тогда композиция $P_1 \circ P_2$ транзитивных отношений, не транзитивна.
	
	\textbf{Ответ:} не обязательно транзитивна.
	
	\section {Задача 3}
	
	Путь $a$ и $b$ - карты из колоды. Отношение заданное на колоде $R = $ <<Одна из карт старше 10-ки, другая младше>>. Тогда $a R b \Leftrightarrow (a < 10$ и $b > 10)$  или $(a > 10$ и $b < 10)$. 
	
	Поймем, что $R$ - симметрично. Действительно если $a R b$, то и $b R a$ (условие - одна из карт больше 10, другая - меньше, выполняется вне зависимости от порядка карт).
	
	Так как одна карта может быть больше 10, меньше 10 или же равна 10, то не для каких карт $a$ в колоде не верно $a R a$, значит $R $ - не рефлексивно.
	
	Поймем также, что $R$ не транзитивно. Например, если рассмотреть карты $6$, $11$, $5$, то получим $6 R 11$, $11 R 5$, но неверно, что $6 R 5$. 
	
	Посчитаем количество пар карт $(a, b)$, таких, что $a < 10 < b$. Карт, младших 10 - всего $4 \cdot 4 = 16$ (по четыре в каждой масти). Карт, старше чем 10-ка так же 16 (4 в каждой масти). По правилу произведения кол-во пар $(a, b)$ равно $16 \cdot 16 = 256$. Но так как $R$ - симметричное отношение, то раз $a R b$, то и $b R a$. Тогда $|R| = 256 \cdot 2 = 512$.
	
	\textbf{Ответ:} 512.
	
	\section{Задача 4}
	
	\subsection {а)} 
	Да, это отношение может быть рефлексивным. Например, рассмотрим Декартово произведение этого множества с собой без трех пар вида $(a, a)$. Получим подмножество Декартова произведения, которое задает какое-то симметричное отношение из 33 пар.
	
	\textbf{Ответ:} да, может.

	\subsection{б)}
	Докажем от противного, что такое отношение не может быть транзитивным. 
	
	Положим, $R$ - транзитивное отношение на множестве $M$ из 6 элементах, такое что $|R| = 33$. Тогда $R$ - подмножество $M \times M$ без трех пар $(x_0, y_0)$, $(x_1, y_1)$, $(x_2, y_2)$. 
	
	Тогда рассмотрим граф отношений, выкинув из него те дуги, которые как-то соединяют $x_i$ и $y_i$, и дуги которые соединяют вершину саму с собой. Оставшиеся дуги обладают рефлексивностью (если дуга ведет из $a$ в $b0$, то есть дуга и из $b$ в $a$), т.е. данный граф соответствует полному графу на 6 вершинах без трех ребер  $G = (E, V)$.
	
	Если в $G$ нет вершины со степенью меньше, чем $3$, то он Гамильтонов граф (выполняется условие Дирака), т.е. в нем есть цикл проходящий по каждой вершине ровно 1 раз. В этом случае поймем, что в силу транзитивности $R$, если из $a$ в $b$ есть путь через $c$, то есть дуга из $a$ в $c$, а так как подмножество  $R^{\prime} \subset R$, которое изображает $G$, симметрично, то есть дуга и из $c$ в $a$. Значит все вершины в графе $G$ соединены ребром, значит он - полный. Противоречие.
	
	Тогда в $G$ есть вершина $u$ со степенью меньше чем 3. Так как $G$ отличается от полного только отсутствием 3-х ребер, то $deg(u) = 2$. Но тогда $\forall v$, $w \in V$, $v \ne u$, $w \ne u$, $w \ne v$: $(v, w) \in E$. Пусть $u$ не соединена с $x$, и соединена c $y$. Значит $(x, y) \in E$, но в силу транзитивности $R$, так как $x R y$, $y R u$, то $x R u$. Аналогично: $u R y$, $y R x \Rightarrow$ $u R x$. Тогда $(x, u) \in E$ - противоречие.
	
	Вывод: $R$ не может быть транзитивным.
	
	\textbf{Ответ:} нет, не может.
	
	\section {Задача 5} 
	
	\subsection {a)}
	 Отношение на множестве $A$ задается подмножеством пар из $ A \times A$. $|A \times A| = n^2 $, тогда всего подмножеств в $A \times A$: $2^{n^2}$. Соответственно и число всех бинарных отношений на множестве $A$ равно $2^{(n^2)}$.
	 
	 \textbf{Ответ:} $2^{(n^2)}$. 
	 
	 \subsection {б)}
	 Если $R$ - рефлексивно, то все пары вида $(a, a) $ ($n$ штук) $ \in A \times A$ содержатся и в $R$, а остальные пары могут как содержаться, так и не содержаться. Т.е. $|R| = 2^{(n^2 - n)}$.
	 
	 \textbf{Ответ:} $2^{(n^2 - n)}$.
	  
	\subsection {в)}
	Если $R$ - симметрично из $(a, b) \in R \Rightarrow (b, a) \in R$. Тогда посчитаем число $x$ таких подмножеств из $A \times A$ в которых одновременно содержатся и $(a, b)$ и $(b, a)$, $a \ne b$. 
	
	Для каждой пары $(a, b)$ есть выбор включить ее и пару $(b, a)$ в подмножество или нет; число таких пар $(a, b)$, где $a \ne b$ равно $n(n - 1)$; причем если включить $(a, b)$ в отношение, то необходимо включить и $(b, a)$, значит число таких отношений $x = 2^{\frac{n(n - 1)} {2}}$. Осталось учесть, что $R$ может содержать пары вида $(a, a)$. Тогда в каждое из уже посчитанных подмножеств мы можем добавить от $0$ до $n$ новых пар, не меняя симметричности задаваемых ими отношений, т.е. дополнительно нужно выбрать включать какие-то из этих $n$ пар в отношение или нет. Это и дает нам число симметричных отношений: $2^{\frac{n(n - 1)} {2} + n}$.
		
	\textbf{Ответ:} $2^{\frac{n(n - 1)} {2} + n}$.
	
	  
	\subsection {г)}
	Поймем, что искомая оценка - это число всех подмножеств  - $x$ из прошлого пункта. И правда, множество $R$ - антисимметрично если $a R b$ и $b R a$ влечет $a = b$, т.е. число антисимметричных отношений - в точности число отношений, в которых пары $(a, b)$ и $(b, a)$ не входят одновременно, где $a \ne b$, т.е. $2^{(n^2)} - 2^{\frac{n(n - 1)} {2}}$.
	
	\textbf{Ответ: } $2^{(n^2)} - 2^{\frac{n(n - 1)} {2}}$.
	
	\section {Задача 6}
	\subsection {а)}
	Из того, что $P$ - отношение эквивалентности, значит оно транзитивно, рефлексивно и симметрично. Тогда понятно, что ${a, b, c}$ - принадлежат одному классу эквивалентности, а элементы ${d, e}$ - другому, т.к. в силу транзитивности отношения $P$ все $a, b, c$ попарно эквивалентны, и не эквивалентны элементу $d$, который в свою очередь эквивалентен $d$.
	
	Рассмотрим граф отношений $R$ на данном множестве.
	
	\includegraphics[scale=0.5]{6_1.png}
	
	Так как $e \bar{P} f$, то $e$ и $f$ не эквивалентны. С другой стороны про отношения $f$ с $a, b, c$ ничего не известно, кроме того, что возможно $f$ принадлежит тому же классу эквивалентности.
	
	Так как отношение эквивалентности определяется разбиением множества на непересекающиеся подмножества возможно всего 2 варианта отношения $P$: первый, когда $P$ разбивает множество $A$ на два класса эквивалентности $\{a, b, c, f\}$ и $\{e, d\}$, второй, когда классов эквивалентности 3 $\{a, b, c\}$, $\{f\}$ и $\{e, d\}$.
	
	\subsection {б)}
	
	Для множества $A$, содержащего дополнительный элемент $g$ выпишем все возможный разбиения на классы эквивалентности, с учетом пункта $a)$.
	
	$$\{a, b, c, f, g\} \quad \{b, e\}$$
	$$\{a, b, c, f\} \quad \{b, e, g\}$$
	$$\{a, b, c, f\} \quad \{b, e\} \quad \{g\}$$
	
	$$\{a, b, c, g\} \quad \{b, e\} \quad \{f\}$$
	$$\{a, b, c\} \quad \{b, e, g\} \quad \{f\}$$
	$$\{a, b, c\} \quad \{b, e\} \quad \{f, g\}$$
	$$\{a, b, c\} \quad \{b, e\} \quad \{f\} \quad \{g\}$$
	
	Т.е. для каждого из классов эквивалентности для каждого из двух вариантов $P$ из прошлого пункта, $g$ может принадлежать одному из них или образовывать отдельный класс.
	
	Каждое из приведенных разбиений на непересекающиеся множества задает возможное отношение эквивалентности $P$.
	
	\section{Задача 7}
	
	Докажем формулу $p_{n+1} = \sum_{i=0}^{n} {n \choose i} p_i$ по индукции по $n$.
	
	База. Число отношений $p_0$ эквивалентности на множестве $A_0 = \emptyset$ равно способу разбить $A_0$ на непересекающиеся множества (здесь и далее будем под разбиением на непересекающиеся множества понимать некое отношение эквивалентности), т.е. $1$. Число отношений эквивалентности $p_1$ на множестве $A_1 = {a}$ равно числу способов разбить $A_1$ на непересекающиеся подмножества, т.е. $1$. В свою очередь: $1 = {0 \choose 0} p_0$.
	
	Предположение, пусть формула верна для ${0, \cdots, n}$.
	
	Переход ${0, \cdots, n} \Rightarrow n + 1$.
	Рассмотрим множество $B$ из $n + 1$ элемента. Зафиксируем какой-то один элемент $x \in B$. Рассмотрим множество $B^{\prime} = B \subset \{x\}$. Любое подмножество $B^{\prime}$ из $k < n$ элементов можно разбить на $p_k$ непересекающихся подмножеств. Рассмотрим тогда такое разбиение $B$ на подмножества:
	
	$$\{a_1, \cdots,a_k\} \quad \{x, a_{k + 1},\cdots, a_{n}\}, \quad (a_i \in B^{\prime})$$
	
	Поймем, что все разбиения $B$ на подмножества, в которых $x$ входит в подмножество из $n - k$ элементов (не считая сам $x$)  могут быть получены разбиением $ \{a_1, \cdots,a_k\} $ на другие непересекающиеся подмножества ровно $p_k$ способами (по предположению индукции). А так как в любом разбиении $B$ на непересекающиеся подмножества $x$ входит в подмножество какой-то длины $n - k$, то нам достаточно найти способов разбить $B^{\prime}$ на подмножества $C$ и $D$ длины $n - k$ и $k$, а $D$ разбить вообще на все возможные разбиения на подмножества.
	
	Выбрать $k$ элементов из $B^{\prime}$, с которыми $x$ будет образовывать подмножество $B$, можно ${n \choose n - k} = {n \choose k} $ способами. Разбить оставшиеся $n - k$ элементов на непересекающиеся подмножества, по предположению индукции, можно $p_{k}$ способами. Тогда всего разбиений $B$ на непересекающиеся подмножества таких, что $x$ входит в подмножество из $n - k$ элементов равно $p_{k} {n \choose k}$.
	
	Из этих соображений получаем, что общее число разбиений $B$ на непересекающиеся подмножества равно числу способов разбить его так, чтобы $x$ лежал в подмножестве из 0, 1-ого, ... , $n$ элементов (не считая $x$). То есть получаем формулу для $p_{n + 1}$: 
	$$ p_{n+1} = {n \choose 0} p_0 + \cdots + {n \choose n} p_n= \sum_{i=0}^{n} {n \choose i} p_i$$    
\end{document}

