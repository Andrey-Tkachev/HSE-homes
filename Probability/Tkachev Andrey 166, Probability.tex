\documentclass{article}
\usepackage{cancel}
\usepackage[utf8]{inputenc}
\usepackage {titlesec}
\usepackage[english,russian]{babel}
\usepackage{amssymb}
\usepackage{amsmath}
\usepackage{graphicx}
\usepackage{numprint}
\graphicspath{{pictures/}}

\titlespacing*{\section}{\parindent}{*4}{*4}

\title{Домашнее задание 12}
\author{Ткачев Андрей, группа 166}
\date{\today}
\newcommand{\niton}{\not\owns}
\newcommand{\pr}{^{\prime}}
\newcommand{\ppr}{^{\prime\prime}}
\newcommand{\xp}{x^{\prime}}
\newcommand{\xpp}{x^{\prime\prime}}
\newcommand{\xppp}{x^{\prime\prime\prime}}
\newcommand{\pair}[2]{(#1,\ #2)}
\newcommand{\andi}{$ и $}
\newcommand{\half}[1]{\frac{#1}{2}}

\begin{document}
	\maketitle
	\paragraph{Задача 1.} Пусть $\Omega$ - множество всех исходов, а $A$ - множество исходов, при которых отношение транзитивно. $|\Omega| = 2^4 = 16$ - в бинарном отношении на двух элементах элемент может соотноситься с собой, а может и нет, может вступать в отношение с другим элементом, как левый операнд, может и не вступать; симметрично для второго элемента. 
	
	Посчитаем $|\bar{A}|$ - количество не транзитивных бинарных отношений на множестве из двух элементов. Пусть элементы $a,\ b$; отношение $R$. Поймем, что транзитивность нарушается только в трех случаях: когда $aRb$, $bRa$, $\pair{a}{a} \notin R$ и $bRb$; когда $bRa$, $aRb$, $\pair{b}{b} \notin R$ и $aRa$;
	или когда $aRb$, $bRa$, $\pair{a}{a} \notin R$ и $\pair{b}{b} \notin R$. Таким образом,  $|\bar{A}| = 3$.

	Так как модель равновероятная, то $Pr[A] = \frac{|A|}{|\Omega|} = \frac{|\Omega| - |\bar{A}|}{\Omega} = \frac{16 - 3}{16} = \frac{13}{16}$
	
	\subparagraph{Ответ:} $\frac{13}{16}$.
	
 	\paragraph{Задача 2.} Пусть $\Omega$ - множество всех всюду определенных функций $f$ из $A$ в $B$ (где $|A| = a,\ |B| = b$). $|\Omega| = b^a$ (Для каждого элемента из $A$ можно сопоставить элемент из $B$ $b$ способами). 
 	
 	Пусть $M$ - множество тех исходов, при которых $f$ - биекция. Посчитаем $|M|$: $|M| = \prod_{i=1}^{max(a,\ b)} (min(a,\ b) - i + 1)$ (Если $f$ - биекция, то каждому $x \in A$ взаимно однозначно соответствует $y \in B$, а значит мощности множеств равны, т.е. $a = b$; В случае равенства множеств выберем первому эл-ту из $A$ $b$ способами эл-т из $B$, второму - $b - 1$ способом и тд. т.е. всего $b!=a!$ биекций; В случае $a \ne b$ одно из чисел $a, b$ больше другого, а значит произведение $\prod_{i=1}^{max(a,\ b)} (min(a,\ b) - i + 1)$ содержит множителем 0, значит равно 0, что отражает число биекций в этом случае).
 	
 	Модель равновероятная, значит $Pr[M] = \frac{|M|}{|\Omega|} = \frac{\prod_{i=1}^{max(a,\ b)} min(a,\ b) - i + 1}{b^a}$.
 	
 	\subparagraph{Ответ:}  $\frac{|M|}{|\Omega|} = \frac{\prod_{i=1}^{max(a,\ b)} (min(a,\ b) - i + 1)}{b^a}$.
 	
 	\paragraph{Задача 3.}
 	Посчитаем отдельно те исходы, при которых в перестановке $37$ стоит на $19$ месте, а $36$ - среди первых $18$-ти ($A$) и те исходы, при которых $37$ находится среди первых 18 чисел ($B$). Поймем, что события $A$ и $B$ не пересекаются и в объединении дают событие $X$ <<Наибольшее число среди первых 18 больше наибольшего среди 18 последних>> (Действительно, наибольшее число в перестановке - 37, значит <<побеждают>> те 18 чисел, которые содержат его. В случае когда 37 не входит ни в первые ни в последние 18 чисел, <<выигрывают>> те 18 чисел, среди которых есть 36).
 	
 	$|A| = 18 \cdot 35!$ - $18$ способами размещаем число 36 среди первых 18 и $35!$ - расставляем оставшиеся числа. $|B| = 18 \cdot 36!$ - размещаем одним из 18 способов 37 среди первых 18 и дополняем перестановку всеми возможными способами.
 	
 	Так как всего перестановок на 37 элементах $37!$ и $A$, $B$ - равновероятные события, то $Pr[X] = \frac{|X|}{37!} = \frac{|A| + |B|}{37!} = \frac{18(1 + 36)}{36\cdot37} = \frac{1}{2}$
 	
 	\subparagraph{Ответ:} $\frac{1}{2}$.
 	
 	\paragraph{Задача 4.}
 	Посчитаем множество $A$ исходов, при которых последовательность заканчивается на 1. $(a_1,\ a_2,\ a_3,\ a_4,\ a_5) \in A$, если $a_5 = 1$, $37 > a_1 > a_2 > ... > a_4 > 1$
 	
 	Т.е. $a_1,\ ...\ a_4$ - как-то упорядоченные числа от 2 до 36. Поймем, что любые четыре попарно различных числа можно поставить в порядке убывания, значит таких наборов из 4-х чисел - число способов выбрать 4 каких-то числа от 2-х до 36: ${36 - 1 \choose 4}$. Тогда $|A| = {35 \choose 4}$.
 	
 	Всего же убывающих последовательностей длины 5 с числами от 1 до 36: ${36 \choose 5}$. Тогда $Pr[A] = \frac{{35 \choose 4}}{{36 \choose 5}} = \frac{5}{36}$.
 	
 	\subparagraph{Ответ:} $\frac{5}{36}$.
 	
 	\paragraph{Задача 5.}
 	Посчитаем число неубывающих последовательностей $\linebreak (x_1,\ x_2,\ x_3,\ x_4,\ x_5)$ где $1 \leqslant x_i \leqslant 36$. Взаимно однозначно сопоставим каждому набору  $(x_1,\ ...,x_5)$  набор $(x_1 - 1,\ x_2 - x_1,\ x_3 - x2,\ ..., 36 - x_5) = (d_1,\ d_2,\ d_3,\ d_4,\ d_5,\ d_6)$ (Каждый набор из $x$ однозначно задает $d$, а по набору $(d_i)$ можно однозначно восстановить $(x_i)$). Тогда число наборов $(d_i)$ равно числу наборов $(x_i)$. Заметим, что $d_1 + ... + d_6 = 35$, при этом $0 \leqslant d_i$. Т.е. мы приходим к задаче Муавра. Число таких разбиений 35 на слагаемые: ${35 + 6 - 1 \choose 5} = {40 \choose 5}$.
 	
 	Поймем, что  неубывающие последовательности длины 5, начинающиеся с 1-цы получаются приписыванием к 1-це неубывающую последовательность длины 4. Неубывающих последовательностей длины 4: ${35 + 5 - 1 \choose 4} = {39 \choose 4}$ - по аналогии с случаем большей размерности. Т.е. если $A$ - событие <<последовательность начинается с 1>> вероятностного пространства неубывающих последовательностей чисел от 1 до 36, то $|A| = {39 \choose 4}$. 
 	
 	Так как все исходы равновероятны, то $Pr[A] = \frac{|A|}{{40 \choose 5}} = \frac{{39 \choose 4}}{{40 \choose 5}} = \frac{5}{40} = \frac{1}{8}$
 
 	\subparagraph{Ответ:} $\frac{1}{8}$.
 	
 	\paragraph{Задача 6.}
 	Вероятностное пространство $\Omega$ - двоичные слова длины 21, все исходы равно возможны, $|\Omega| = 2^{21}$. Искомая вероятность - $Pr[X]$
 	
 	Примем за $\alpha$ - число единиц в первых 10 битах, $\beta$ - число единиц в последних 10 битах, $\gamma$ - в последних 11 битах.
 	
 	Посчитаем вероятность события $A$ - <<$\alpha = \beta$, 10-ый бит - 1 (нумерация с 0)>>. В располагаться $\alpha$ единиц могут ${10 \choose \alpha}$ способами в начале и независимо ${10 \choose \beta} = {10 \choose \alpha}$ способами в конце. Т.е. $|A| = \sum_{i=0}^{10} {10 \choose i} \cdot {10 \choose i} = {10 \choose i} \cdot {10 \choose 10 - i} = {20 \choose 10}$ (это равенство следует из доказанного в ДЗ-3; установить 10-ый бит в 1 можно ровно 1-м способом). $Pr[A] = \frac{{20 \choose 10}}{|\Omega|}$.
 	
	Посчитаем вероятность события $B$ - <<$\alpha < \beta$>>. Поймем, что если событие $C$ - <<$\alpha > \beta$>>, то $Pr[B] = Pr[C]$ из соображений симметрии - длины последовательностей бит равны, все последовательности равно вероятны. Пусть тогда  событие $D$ - <<$\alpha = \beta$>>, по аналогии с $|A|$ - $|D| = 2\cdot{20 \choose 10}$ (теперь 10-ому биту позволено быть и 0, и 1). Поймем, что $Pr[B] + Pr[C] + Pr[D] = 1$, т.к. эти события описывают все вероятностное пространство. Тогда $2 \cdot Pr[B] = 1 - Pr[D] = 1 - \frac{2 \cdot {20 \choose 10}}{|\Omega|}$, $Pr[B] = \frac{1}{2} - \frac{{20 \choose 10}}{|\Omega|}$.
	
	Поймем, что $Pr[B] + Pr[A] = Pr[X]$ - т.к. события $A$ и $B$ не пересекаются, а в объединении описывают все исходы, при которых $\gamma > \alpha$. Тогда $Pr[X] = \frac{{20 \choose 10}}{|\Omega|} + \frac{1}{2} - \frac{2 \cdot {20 \choose 10}}{2|\Omega|} = \frac{1}{2}$.
	
	\subparagraph{Ответ:} $\frac{1}{2}$.
	
	\paragraph{Задача 7.}
	Вероятностное пространство $\Omega$ - все перестановки карт в колоде из 36 карт; все исходы равновероятны; $|\Omega| = 36!$. Пусть событие $A_x$ - <<среди первых $x$ карт есть хотя бы один туз>>.Нужно найти минимальный $x$, такой, что $Pr[A_x] > \frac{1}{2}$, что мы и сделаем.
	
	Поймем, что $Pr[A_x] > \frac{1}{2} \Leftrightarrow Pr[\bar{A_x}] \leqslant \frac{1}{2}$ (событие $\bar{A_x}$ <<Среди первых $x$ карт нет тузов>> и событие $A_x$ в объединение дают все пространство). Посчитаем $|\bar{A_x}|$. Среди первых $x$ карт нет, значит все тузы расположены среди $36 - x$ последних карт, причем остальные карты располагаются как угодно. Таких перестановок колоды $({36 - x \choose 4} \cdot 4!) \cdot 32!$ - (выбираем 4 места под тузы в колоде, расставляем тузов $4!$ способами и на оставшиеся $32$ места прочие карты всеми возможными способами). Таким образом $|\bar{A_x}| = ({36 - x \choose 4} \cdot 4!) \cdot 32! = \frac{(36 - x)!32!}{(32 - x)!}$. $Pr[|\bar{A_x}|] = \frac{(36 - x)!32!}{(32 - x)!|\Omega|} = \frac{(36 - x)!32!}{(32 - x)!36!} = P(x)$.
	
	Найдем наименьшее в натуральных числах решение \linebreak неравенства $\frac{(36 - x)!32!}{(32 - x)!36!} \leqslant \frac{1}{2}$. Для начала поймем, что если $P(a) > \frac{1}{2}$, то $\forall a\pr < a:\ P(a\pr) > \frac{1}{2}$, так как $P(a) > P(a\pr)$ (Действительно, вероятность не встретить туза среди первых $a\pr$ карт больше, чем встретить туза среди первых $a$ карт, т.к. $\bar{A}_a \subseteq \bar{A}_{a\pr} $). Заметим, что $P(6) = \frac{87}{187} \leqslant \frac{1}{2}$, $P(5) = \frac{899}{1683} > \frac{1}{2}$. Тогда $x = 6$ - минимальное решение неравенства.
	
	Получается из колоды карт необходимо вытащить минимум 6 карт, чтобы вероятность встретить туза была больше $\frac{1}{2}$.
	\subparagraph{Ответ:} 6.
	
	\paragraph{Задача 8.}
	Вероятностное пространство $\Omega$ - все возможные наборы дней рождений (без 29-ого февраля) в группе из 30 человек. $|\Omega| = 364^{30}$. Событие $A$ - <<в наборе найдется две совпадающие даты>>. Тогда $\bar{A}$ - <<Все даты в наборе из 30 попарно различны>>. Поймем, что $A$ и $\bar{A}$ - взаимно дополняющие события и $Pr[A] + Pr[\bar{A}] = 1$.
	
	Найдем $Pr[\bar{A}]$. Посчитаем число попарно различных наборов чисел от 1 до 364 из 30 элементов: $|\bar{A}| = {364 \choose 30} \cdot 30!$ - для каждого множества нужных дат, мощностью 30 введем порядок $30!$ способами, получив все возможные наборы попарно различных дат. $Pr[\bar{A}] = \frac{\bar{A}}{|\Omega|} = \frac{364!}{334!364^{30}}$.
	
	Тогда $Pr[A] = 1 - \frac{364!}{334!364^{30}} = 1 - \frac{335 \cdot ... \cdot 364} {364^{30}} \approx 0.707331$ в чем можно убедиться произведя соответствующие вычисления.
	
	\paragraph{Задача 9.}
	Всего турниров на $n$ элементном множестве - $2^{\half{n(n - 1)}}$ (всего всего пар элементов в отношениях - $\half{n(n - 1)}$, причем между $\pair{a}{b}$ дуга отношения может идти либо от $a$ к $b$, либо от $b$ к $a$ - 2 опции, для каждой пары отношений). Тогда, если $\Omega$ - вероятностное пространство из условия, то $|\Omega| = 2^{\half{n(n - 1)}} $. Линейных порядков на $n$ элементном множестве - $n!$ - по количеству способов упорядочить $n$ элементов. Тогда вероятность события $A$: случайный турнир - линейный порядок, равна $Pr[A] = \frac{|A|}{|\Omega|} = \frac{n!}{2^{\half{n(n - 1)}}}$.
	
	Посчитаем $\lim\limits_{x \rightarrow \infty} \frac{n!}{2^{\half{n(n - 1)}}}$.
	
	$$\frac{n!}{2^{\half{n(n - 1)}}} < \frac{n!}{2^{(\half{(n - 1)}) ^ {(n - 1)}}} = 1 \cdot \frac {2}{2^{\half{(n - 1)}}} \cdot \frac {3}{2^{\half{(n - 1)}}} \cdot ... \cdot \frac{n} {2^{\half{(n - 1)}}}$$

	Докажем теперь, $\frac{n} {2^{\half{(n - 1)}}} < \half{1}$. 
	Заметим, что последовательность $a_n = \{\frac{n} {2^{\half{(n - 1)}}} \}$ - убывает: $\frac{a_{n + 1}}{a_n} = \frac{\frac{n + 1} {2^{\half{n} } } } {\frac{n} {2^{\half{n - 1} } } } = \frac{n + 1}{\sqrt{2}n} < 1$, $a_{11} = \frac{11} {2^{5}} < \half{1}$ $\Rightarrow$ $a_n < \half{1},\ 11 \leqslant n$. Но тогда при больших $n$: $\prod_{i=2}^{n} \frac {i}{2^{\half{(n - 1)}}} < \prod_{i=2}^{n} \frac {n}{2^{\half{(n - 1)}}} < (\half{1})^{n - 1}$.
	
	Тогда имеем $1 \cdot \frac {2}{2^{\half{(n - 1)}}} \cdot \frac {3}{2^{\half{(n - 1)}}} \cdot ... \cdot \frac{n} {2^{\half{(n - 1)}}} < (\half{1})^{n - 1}$. Т.е. $0 \leqslant \frac{n!}{2^{\half{n(n - 1)}}} \leqslant (\half{1})^{n - 11}$ при $n \rightarrow \infty$, но $\lim\limits_{n \rightarrow \infty}0 = \lim\limits_{n \rightarrow \infty}  (\half{1})^{n - 1} = 0$. Тогда по теореме о двух милиционерах: $\lim\limits_{n \rightarrow \infty} \frac{n!}{2^{\half{n(n - 1)}}} = 0$.
	
	Таким образом вероятность того, что турнир окажется линейным порядком стремится к нулю, при $n \rightarrow \infty$.
 \end{document}