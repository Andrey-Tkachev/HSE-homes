\documentclass{article}

\usepackage {titlesec}
\usepackage[utf8]{inputenc}
\usepackage[english,russian]{babel}

% Offset setup %
\usepackage[left=15mm,
            top=15mm, 
            right=10mm,
            bottom=15mm, nohead, nofoot]{geometry}

% Maths packages %
\usepackage{amssymb}
\usepackage{amsmath}

% Special symbols %
\usepackage{wasysym}
\usepackage{cancel}
\usepackage{graphicx}
\usepackage{numprint}
\graphicspath{{pictures/}}

\titlespacing*{\section}{\parindent}{*4}{*4}



\title{Домашнее задание 19}
\author{Ткачев Андрей, группа 166}
\date{\today}

% Other%
\newcommand{\pr}{^{\prime}}
\newcommand{\ppr}{^{\prime\prime}}
\newcommand{\xp}{x^{\prime}}
\newcommand{\xpp}{x^{\prime\prime}}
\newcommand{\xppp}{x^{\prime\prime\prime}}
% Alias %
\newcommand{\pair}[2]{(#1,\ #2)}
\newcommand{\andi}{$ и $}
\newcommand{\xor}{\oplus}
%\newcommand{\xor}{\oplus}%

% Fracs %
\newcommand{\half}[1]{\frac{#1}{2}}
% Pretty Num letters%
\newcommand{\N}{\mathbb{N}}
\newcommand{\R}{\mathbb{R}}
\newcommand{\Q}{\mathbb{Q}}
\newcommand{\M}{\mathbb{M}}
\newcommand{\conti}{2^{\N}}

% Vector graphics %
\usepackage{tikz}
\definecolor{white}{rgb}{1.0, 1.0, 1.0}
\definecolor{black}{rgb}{0.0, 0.0, 0.0}
\definecolor{royalblue}{RGB}{255,170,128}
\definecolor{lgr}{RGB}{168, 228, 160}
\definecolor{lblue}{RGB}{24,205,255}

\usetikzlibrary{decorations.markings}
\usetikzlibrary{shapes.geometric}

\pgfdeclarelayer{edgelayer}
\pgfdeclarelayer{nodelayer}
\pgfsetlayers{edgelayer,nodelayer,main}

\tikzstyle{none}=[inner sep=0pt]
\tikzstyle{node}=[circle,fill=white,draw=black,thick]

\begin{document}
    \maketitle
    \paragraph{Задача 1.}
    Здесь и далее для упрощения восприятия будем строить схему не для $f:\{0, 1\}^{{n \choose 2}} \rightarrow \{0, 1\}$, а для $f:\{0, 1\}^{n^2} \rightarrow \{0, 1\}$, т.е. схему принимающую на вход матрицу смежности задающую граф (т.е. вход будет содержать дублирующиеся данные, т.к. матрица смежности обычного графа симметрична относительно главной диагонали). Тогда вход схемы --- переменные $x_{ij}$, где $x_{ij} = 1$, если вершины $i$ и $j$ соединены ребром, и $x_{ij} = 0$ в противном случае.

    Матрица смежности графа, содержащего изоолированные вершины, примечательна тем, что одна из ее строк --- нулевая. Действительно, в изолированную вершину $i$ ребра не ведут, а значит и строка $i$ нулевая (верно и обратное: если есть нулевая строка, то есть и изолированная вершина). Т.е. нам необходимо узнать, есть ли в матрице нулевая строка. Схема, которая это делает:

    \begin{center}
        \begin{tikzpicture}
            \begin{pgfonlayer}{nodelayer}
            \node [style = node] (x00) at (-6, 0) {$x_{11}$};
            \node [style = node] (x01) at (-4, 0) {$x_{12}$};

            \node [] (ldots1) at (-2, 0) {$\ldots$};
            \node [style = node] (x0n) at (0, 0) {$x_{1n}$};

            \node [style = node] (or01) at (-4, -1.5) {$\vee$};
            \node [] (ordots1) at (-2, -1.5) {$\ldots$};
            \node [style = node] (or0n) at (0, -1.5) {$\vee$};
            \node [style = node] (not1) at (2, -1.5) {$\lnot$};

            \node [] (vdots) at (-2, -2) {$\vdots$};

            \node [style = node] (xn0) at (-6, -3) {$x_{n1}$};
            \node [style = node] (xn1) at (-4, -3) {$x_{n2}$};
            \node [] (ldots2) at (-2, -3) {$\ldots$};
            \node [style = node] (xnn) at (0, -3) {$x_{nn}$};

            \node [style = node] (orn1) at (-4, -4.5) {$\vee$};
            \node [] (ordots2) at (-2, -4.5) {$\ldots$};
            \node [style = node] (ornn) at (0, -4.5) {$\vee$};
            \node [style = node] (not2) at (2, -4.5) {$\lnot$};

            \node [style = node, fill=royalblue] (or5) at (4, -2) {$\lor_n$};
            \end{pgfonlayer}

        \begin{pgfonlayer}{edgelayer}
            \draw [->, thick] (x00) to (or01);
            \draw [->, thick] (x01) to (or01);
            \draw [->, thick] (x0n) to (or0n);
            \draw [->, thick, dashed] (or01) to (ordots1);
            \draw [->, thick, dashed] (ordots1) to (or0n);

            \draw [->, thick] (xn0) to (orn1);
            \draw [->, thick] (xn1) to (orn1);
            \draw [->, thick] (xnn) to (ornn);
            \draw [->, thick, dashed] (orn1) to (ordots2);
            \draw [->, thick, dashed] (ordots2) to (ornn);
            
            \draw [->, thick] (or0n) to (not1);
            \draw [->, thick] (ornn) to (not2);

            \draw [->, thick] (not1) to (or5);
            \draw [->, thick] (not2) to (or5);
            \draw [->, thick, dashed] (2, -2.4) to (or5);
        \end{pgfonlayer}
    \end{tikzpicture}
    \end{center}

    Под $\lor_n$ здесь понимается подсхема из каскада дизъюнкций размера $n - 1$, который вычисляет конъюнкцию $n$ аргументов.

    Размер данной схемы $n^2 + (n - 1)^2 + n + (n - 1) = O(n^2)$.

    \paragraph{Задача 2.}
    Пусть вершины $i, j, k$ графа $G$ образуют треугольник. Тогда в матрице смежности графа $G$ в ячейках $a_{ij}, a_{ik}, a_{kj}$ должны стоять единицы. Получается, для того чтобы проверить, есть ли в графе треугольник или нет, необходимо всего лишь узнать есть ли три такие переменные $x_{ij}, x_{ik}, x_{kj}$ на входе, что $i\ne j \ne k \ne i$ и $x_{ij}=x_{ik}=x_{kj}$. Тогда схема будет вычислять формулу $\neg\vee_{i<j<k \le n} (x_{ij}\land x_{ik}\land x_{kj})$. 

    В виде рисунка данная схема выглядит грамоздко, потому довольствуемся ее словесным описанием: каждые два элемента в строке $i$, начиная с элемента $i+1$ (небольшая оптимизация, использующая то, что матрица симетрическая) соединены конъюнкцией между собой, а также с третьим аргументом, стоящем в столбце с индексом второго из множителей и строке с номером первого множителя; все такие блоки конъюнкий объединяются дизъюнкцией, а потом от итоговой дизъюнкции вычислсяется отрицание. Таким образом, если в графе есть трегуольник, то существуют $i<j<k$ (просто упорядоченные номера вершин), для которых конъюнкция истинна, а значит и дизъюнкция истинна, значит результат схемы --- отрицание истины. И наоборот, если треугольников нет, то в любом из слагаемых найдется хотябы один множитель ноль, а значит и дизъюнкция ложна, а схема вычислила истину. Размер схемы пропорционален числу троек чисел $1 \le i < j < k \le n$, которых не больше ${n \choose 3}$, т.е. размер схемы $O(n^3)$.

    \paragraph{Задача 3.}
    Вспомним признак существования эйлерового цикла в связном графе $G(V, E), |V| = n$: граф эйлеров, когда он связен и степени всех его вершин четны. Таким образом, нам необходимо построить схему, определяющую связен ли граф, и проверяющую четность степеней вершин.

    По марице смежности четность вершины $i$ проверяется легко: в $i$-ой строке стоит четное число единиц, а значит сумма $\oplus$ всех элементов строки равна 0. Значит, чтобы проверить, является ли $i$-ая вершина четной, необходимо $n - 1$ элементов сложения $\oplus$ (получается из базисных функций: двух конъюнкций, двух отрицаний и дизъюнкции). Тогда степени всех вершин четны, когда результат всех таких каскадов $\oplus$ (по одному каскаду из $n - 1$ эл-та $\oplus$ на строку) равен единице. Проверка последнего осуществляется за $n - 1$ конъюнкцию. Т.е. проверка степеней на <<эйлеровость>> требует $n(n - 1) + n - 1 = O(n^2)$ элементов. 

    Проверка графа на связность осуществляется путем возведения матрицы смежности в степень $n - 1$ при помощи булева умножения матриц. Элемент булева произведения матриц $A$ и $B$ есть $(A \times B)_{ij} = \lor_{k} (A_{ik}\land B_{kj})$. Согласно теореме о возведнии матрицы смежности в степень $k$, элемент $A^{k}_{ij}$ равен 1, когда существует маршрут из вершины $i$ в вершину $j$ длины $k$, и 0 --- в противном случае. Значит из вершины $i$ есть хоть какой-то путь в вершину $j$, означает, что $\lor{n - 1}_{k=1} A^k_{ij} = 1$. Значит граф связен, когда $\wedge^{n - 1}_{j=1} \lor^{n - 1}_{k=1} A^k_{1j} = 1$ (т.е. существует маршрут, а значит и путь, из вершины 1 в любую другую; максимальная длина пути $n - 1$). Произведение матриц, осущевтсляется через реализацаю формулы привиденной выше --- за $n^2$ конъюнкций и $(n-1)^2$ дизъюнкцию, т.е. за $O(n^2)$ элементов. Тогда возведение в степень $n - 1$ будет стоит $O(n^3)$ элементов схемы (т.е. каждый новый результат --- произведение последней вычесленной матрицы и начальной). Проверка же на связность с уже возведенными матрицами осуществляется за $(n - 1)^2$ дизъюнкицю и $n - 1$ конъюнкцию, т.е. за $O(n^2)$ элементов. Значит, проверка на связность + возведение в степени матрицы смежноси обойдется в $O(n^3)$ элементов.

    Тогда результат работы схемы является конъюнкция результата проверки на четность и результата проверки на связность. 

    \paragraph{Задача 4.}
    Отметим, что условие не совсем верно. Так, например, функцию $f = x \land \bar{x}$ невозможно записать только конъюнкцией и дизъюнкцией, хотя она монотонна (ибо констатнта). Поэтому несколько ослабим утверждение задачи: докажем не для всех монотонных, а ддя монотонных, не константных функций.

    Пусть есть монотонная функция $f$. Пусть она принемает значение 1 на наборе $\vec{x}_i$. Рассмотрим функцию $\displaystyle{g_{x_i} = \land_{x \in \vec{x}_i} x}$. Поймем, что $\forall \vec{x}_0 \ge \vec{x}_i:\ g_{x_i}(\vec{x}_0) = g_{x_i}(\vec{x}_i) = f(\vec{x}_i) = 1$, значит для любых наборов $x_1, \ldots x_n: g_{(x_1, \ldots, x_n)} \le f(x_1, \ldots, x_n)$. Рассмотрим тогда функцию $f_0 = \lor_{f(\vec{x}) = 1} g_{\vec{x}}$. Заметим, что $f_0$ истинна на тех наборах, на которых истинна $f$ и если $f_0$ истинна на каком-то наборе $\vec{x}_0$, то сущесвует такое слагаемое $g_{\vec{x}_1}$, что $\vec{x}_0 \ge \vec{x}_1$, а значит, $f(\vec{x}_1) \ge f(\vec{x}_0) = 1$ по построению $g_{\vec{x}_0}$. Значит $f_0 = f$, так как наборы на которых они принимают значение 1 совпадают.

    Покажем, теперь, что $f_0$ записывается схемой с $O(n2^n)$ элементами. Максимальное число слагаемых в $f_0$ равно ${n \choose 2} + {n \choose 3} + \ldots + {n \choose n} = 2^n - 1$, значит дизъюнкций не больше $2^n$. В каждом слагаемом не более $n$ переменных, а значит и конъюнкций в слагаемом не больше $n$, значит размер схемы не более $n + 2^n - 1 + n(2^n - 1) = O(n2^n)$.

    \paragraph{Задача 5.}
    Оценим сверху число различных схем размером не более $s \ge n$. В базисе $\{\oplus, \cdot, 1\}$ схема является последовательностью функций, каждая из которых есть либо одна из переменных (которых не более $n \le s$), либо одна из базисных функций, примененна не более чем к двум предыдущим элементам схемы.

    Закодируем каждую схему двоичным числом, длиной $s(3 + 2(\lfloor log_2 n \rfloor + 1))$ бит (3 бита на кодирование типа элемента схемы --- одна из 3-х базисных фукнций, переменная или <<ничего>>(длина схемы может быть и меньше $s$); каждый элемент схемы кодируется $\lfloor log_2 n \rfloor + 1$ битами --- своим порядковым номером, значит для базисных функций от двух переменных там будет лежать два числа, для переменных там будет лежать номер переменной, для всего остального --- например нули).

    Данное кодирование инъективно: разным схемам соответствуют разные коды. Значит схем размера $s$ не больше, чем таких последовательностей, т.е. не более, чем $2^{s(3 + 2(\lfloor log_2 n \rfloor + 1))} \le 2^{s(2s + 5)} = O(2^{s^2})$. При $s = n^{100}$ получаем, что схем размера не больше $n^{100}$ ---  $O(2^{n^{200}})$.

    Всего функций размера $n$ --- $2^{2^n}$. Но заметим, что $\displaystyle{ \lim_{n \to \infty} \frac{C_0 \cdot n^{200}}{C_1 \cdot 2^n} = 0}$ ($C$ --- некоторые константы), значит для достаточно больших $n$ число различных функций превосходит число схем размера $n^{100}$, т.е. есть функция, которая не вычислима схемой размера $n^{100}$ или меньше.

\end{document}
