\documentclass{article}
\usepackage{cancel}
\usepackage[utf8]{inputenc}
\usepackage[english,russian]{babel}
\begin{document}

\begin{center}
	\subsection*{Задача 1}
\end{center}
	$$ (2 + 1)^n = {n \choose 0} \cdot 2^n + {n \choose 1} \cdot 2^{n - 1} + ... + {n \choose n} \cdot 2^0 $$
	Заметим, что первые несколько слагаемых образуют неубывающую последовательность. Проверим, как долго будет сохранятся не убывание, решив такое неравенство:
	$$ {n \choose k} \cdot 2^{n - k} < {n \choose k + 1} \cdot 2^{n - k - 1}$$
	И найдем такие $k$ при которых следующее слагаемое больше предыдущего:
	$$ {n \choose k} \cdot 2 < {n \choose k + 1}$$
			$$ \Updownarrow $$
	$$ \frac{n!}{k! \cdot (n - k)!} \cdot 2 < \frac{n!}{(k + 1)! \cdot (n - k - 1)!}$$
			$$ \Updownarrow $$
		$$ (k + 1) \cdot 2 < n - k$$
			$$ \Updownarrow $$
		$$ k < \frac{n- 2}{3} $$
	
	Тогда рассмотрим слагаемое с номером $ k = \lfloor \frac{n - 2}{3} \rfloor + 1 $ (нумерация с 0).
	Оно с одной стороны не меньше всех предыдущих (т.к. до этого $k$ выполнялось неравенство $a_i < a_{i + 1}$, где $a_j$ - какое-то слагаемое  $ \Rightarrow$ при $ k_0 = \lfloor \frac{n - 2}{3} \rfloor + 1 $, $a_{k_0} >= a_i$, $i < k_0$ ), а с другой все последующие слагаемые образуют невозрастающую последовательность, иначе неравенство выполнялось бы и для них.
	\\
	Тогда $a_{\lfloor \frac{n - 2}{3} \rfloor + 1 }$ - максимальное слагаемое в разложении (или одно из максимальных, как при $n = 2$).

\begin{center}
	\subsection*{Задача 2}
\end{center}
Будем считать, что равно-вместительные комнаты одинаковы.

Количество способов выбрать 8 человек, для заселения в 4-х местные квартиры: ${18 \choose 8}$. Количество способов поделить их пополам:
$\frac{ {8 \choose 4}}{2} $ (Т.к. число способов выбрать 4 человека из 8 содержит в себе взаимодополняющие варианты - когда выбираешь сначала 4-х людей, а потом 4-х оставшихся, в то время как 4 выбранных человека однозначно определяют разделение). Т.е. всего способов заселить четырехместные комнаты:
$${18 \choose 8} \cdot \frac{ {8 \choose 4}}{2} $$

Заселить 3-х местные комнаты, с уже заселенными 4-х местными:
$${10 \choose 6} \cdot \frac{ {6 \choose 3}}{2} $$

Заселить 2-х местные комнаты, с уже заселенными 4-х и 3-х местными:
$${4 \choose 4} \cdot \frac{ {4 \choose 2}}{2} $$

Итого способов заселить все комнаты:
$${18 \choose 8} \cdot \frac{ {8 \choose 4}}{2} \cdot {10 \choose 6} \cdot \frac{ {6 \choose 3}}{2} \cdot {4 \choose 4} \cdot \frac{ {4 \choose 2}}{2} $$
$$\Updownarrow$$
$$ \frac{18!}{(4!)^2 \cdot (3!)^2 \cdot  (2!)^2 \cdot 2^3 } $$


\begin{center}
	\subsection*{Задача 3}
\end{center}

\textbf{A)} $ {n \choose m}{m \choose k} $ - это число выбрать подмножества длины $k$ множества $M,\\ |M| = m$, причем $M$ - подмножество $N$, $|N| = n$.\\
Поймем, что посчитать это число можно иначе: сначала выберем $k$ элементов из $N$ (${n \choose k}$ способами), затем дополним их $m - k$ элементами из оставшихся в $N$ (${n - m \choose m - k}$ способами). В итоге мы получим все такие подмножества длины $k$, являющиеся подмножествами множества $M, |M| = m$ и $M$ входит в $N$, а число способов сделать это: ${n \choose k}{n - m \choose m - k}$
$$ \Downarrow $$
$$ {n \choose m}{m \choose k} = {n \choose k}{n - m \choose m - k} $$
\textbf{B)} ${2n \choose n}$ - число способов выбрать $n$ элементных подмножества множества из $2n$ элементов. Поймем, как это можно посчитать по другому. Рассмотрим два множества длиной $n$, объединение которых дает множество из $2n$ элементов. Посчитаем кол-во способов выбрать из этих двух множеств в сумме $n$ элементов. 
Это кол-во равно число способов выбрать из первого множества $0$ элементов, а из второго $n - 0$ + число способов выбрать из первого множества $1$ элементов, а из второго $n - 1$ + ...
\\
Более формально:
	$$ \sum_{i=0}^{n} {n \choose i} \cdot {n \choose n - i}$$
(Т.е. для каждых $i$ элементов первого множества мы выбираем $n - i$ элементов второго)
В силу равенства  $ {n \choose i} = {n \choose n - i} $:
	$$ \sum_{i=0}^{n} {n \choose i} \cdot {n \choose n - i} = \sum_{i=0}^{n} {n \choose i}^2 $$
В силу того, что мы посчитали одно и тоже 2-мя разными способами:
	$$\sum_{i=0}^{n} {n \choose i}^2 =  {2n \choose n} $$
	
\begin{center}
	\subsection*{Задача 4}
\end{center}

Пусть человеку $i$  достанется $x_i$ акций. Тогда количество способов поделить акции равно количеству решений уравнения:
$$ \sum_{k=0}^{4} x_k = 100, x_k >= 1$$
Поймем, что $\sum_{k=0}^{4} x_k = 100, x_k >= 1 \Leftrightarrow \sum_{k=0}^{4} x_k = 95, x_k >= 0$. Последнее уравнение есть задача Муавра, и количество решений такого уравнения:
$$ {95 + 5 - 1 \choose 4} = {99 \choose 4}$$ 
\\
\begin{center}
	\subsection*{Задача 5}
\end{center}
Возьмем какой-нибудь порядок из шести человек и попробуем записать их к врачу именно в этом порядке (т.е. чтобы каждый следующий пациент доктора соответствовал следующему в порядке). Тогда пусть в $i$ день доктор примет $x_i$ человек. Тогда кол-во расписаний для этого порядка людей соответствует числу решений уравнения:
$$ x_0 + x_1 + x_2 + x_3 + x_4 + x_5 = 6 $$
Это задача Муавра, ответ на которую, в данном случае: $ {6 + 5 - 1 \choose 4} = {10 \choose 4}$.
Тогда, т.к. всего порядков пациентов $6!$, вариаторов расписаний:
 $$ {10 \choose 4} \cdot 6!$$
 
 \begin{center}
 	\subsection*{Задача 6}
 \end{center}
 
 \textbf{A)} Каждый человек может проголосовать за $n$ кандидатов, при любом выборе остальных людей. Следовательно всего вариантов распределения голосов, при которых важен голос каждого человека персонально: 
 $$ \prod_{1}^{n} n = n^n $$ 
\textbf{B)} Пусть за $i$ кандидата проголосовало $x_i$ человек. Тогда всего вариантов распределения голосов соответствует числу решений уравнения:
$$ x_0 + x_1 + .. + x_n = n $$
Т.к. $x_i$ - любое (ведь можно голосовать и за себя), то число решений этого уравнения, как следует из задачи Муавра: 
$$ {2n - 1 \choose n} $$

 \begin{center}
 	\subsection*{Задача 7}
 \end{center}
 Пусть:
 \begin{itemize}
 	\item{}$ A $ - "Включен свет" 
 	\item{}$ B $ - "Играет музыка"
 	\item{}$ C $ - "Идет дождь"
\end{itemize}
Пересечение этих трех событий минимально тогда, когда максимально объединение противоположных событий, т.е. тех при которых не играет музыка ИЛИ/И не идет дождь ИЛИ/И не включен свет. Т.е. когда $|\bar{A} \cup \bar{B} \cup \bar{C}| = max$.
 \begin{itemize}
 	\item{}$ |\bar{A}| = 100 - |A| = 20 $ 
 	\item{}$ |\bar{B}| = 100 - |B| = 10 $
 	\item{}$ |\bar{C}| = 100 - |C| = 50 $
 \end{itemize}
 Тогда, максимум $|\bar{A} \cup \bar{B} \cup \bar{C}| = 80$.
 Соответственно, музыка свет и дождь активны одновременно вне этих $80\%$, т.е. в $20\%$ времени минимум.
 
 \begin{center}
 	\subsection*{Задача 9}
 \end{center}
 \textbf{A)} Посчитаем количество таких разбиений, где $x_1 > 3$. Их ровно столько, сколько разбиений числа $6$ на $4$  слагаемых (т.е. если сводить задачу к задаче Муавра, будем считать, что в ящик $x_1$ мы уже положили 4 камушка, теперь осталось расположить оставшиеся 6), а именно:
 $$ {6 + 3 \choose 3} = 84 $$
 Всего же способов разбить $10$ на $ 4 $слагаемых $ {10 + 4 - 1 \choose 10} = 286 $.
 Тогда количество способов разбить $10$ на $ 4$ слагаемых, первое из которых $<= 3$:
 $$ 286 - 84 = 202 $$
 \textbf{B)} Пусть $ |A| $ - число способов разбить $10$ на слагаемые, где $ x_1 > 3 $, $ |B| $, аналогично, но $ x_2 > 3 $. Посчитаем количество способов разбить 10 на слагаемые, в которых хотя бы одно из первых 2-х числе $>3$ (т.е. те разбиения, которые дополняют искомые до всех возможных) по формуле включений и исключений:
 $$ |A \cup B| = |A| + |B| - |A \cap B|$$
 $|A \cap B|$ - число таких разбиений, где и $x_1 >= 4$ и $x_2 >= 4$, т.е. $ |A \cap B| = {2 + 4 - 1 \choose 2} = 10$. (Т.е. число разбиений где $x_1 >= 4$ и $x_2 >= 4$, равно числу разбиений 2-ки на четыре слагаемых, см. выше)
 При этом $ |A| = |B|  $, как симметричные относительно переменных.
  $$ |A \cup B| = |A| + |B| - |A \cap B| = 84 + 84 - 10 = 158$$
 Так, как мы ищем ровно противоположные разбиения, в которых $ x_1, x_2 <= 3$, мы вычтем это число из общего числа разбиений:
 $$ 286 - 158 = 128 $$
 \textbf{C)} Пусть $ |A| $ - число способов разбить $10$ на слагаемые, где $ x_1 > 3 $, $ |B| $, аналогично, но $ x_2 > 3 $, $ |C| $, так же, но $ x_3 > 3 $. По формуле включений и исключений посчитаем разбиения в которых хотя бы одно из первых 3-х чисел больше 3:
 $$ |A \cup B \cup C| = |A| + |B| + |C| - |A \cap B| -  |A \cap C| -  |C \cap B| +  |A \cap B \cap C|$$
 Т.к. уравнение симметрично, относительно переменных, то $|A \cap B| = |C \cap B| = |A \cap C| = 84$. Поймем, что $3$ слагаемых одновременно не могут быть больше трех, иначе выражение в левой части равенства будет хотя бы $12$, а справа $10$, что невозможно. $ \Rightarrow |A \cap B \cap C| = 0$. Тогда:
 $$  |A \cup B \cup C| = 84 \cdot 3 - 10 \cdot 3 + 0 = 222$$
 Так, как мы ищем ровно противоположные разбиения, в которых $ x_1, x_2, x3 <= 3$, мы вычтем полученное число из общего числа разбиений с целью добыть ответ:
 $$ 286 - 222 = 64 $$
 \textbf{D)} Повторим рассуждения первых трех пунктов, добавив лишь $ |D| $, по аналогии, число способов разбить на слагаемые, с последним из них, большим 3-ки. 
  $$ |A \cup B \cup C \cup D| = {4 \choose 1} |A| - {4 \choose 2} |A \cap B| + {4 \choose 3} |A \cap B \cap C| - {4 \choose 4} |A \cap B \cap C \cap D|$$
  $$\Updownarrow$$
  $$ |A \cup B \cup C \cup D| = 4 \cdot |A| - 6 \cdot |A \cap B| + 4 \cdot |A \cap B \cap C| - 1 \cdot |A \cap B \cap C \cap D|$$
  $$\Updownarrow$$
  $$ |A \cup B \cup C \cup D| = 336 - 60 = 276 $$
  Так, как мы ищем ровно противоположные разбиения, в которых $ x_1, x_2, x3, x4 <= 3$, мы вычтем полученное число из общего числа разбиений с целью добыть ответ:
  $$ 286 - 276 = 10 $$
 
 
 
 \begin{center}
 	\subsection*{Задача 10}
 \end{center}
	 Будем считать, что строчки нумеруются с $0$.
 Тогда $k$-ое число $n$-ой строчки треугольника Паскаля имеет вид ${n \choose k}$.
 
 Заметим, что после нечетной строчки треугольника П. идет строчка, в которой все числа, кроме первого и последнего четны. Причина этому рекуррентная модель поведения: ${n \choose k} = {n - 1 \choose k} + {n - 1 \choose k - 1}$.
 Тогда поймем, в каких строчках все числа, кроме первого и последнего четны. Т.е. при каких $n$, ${n \choose k}$ кратно $2$, $k \cancel{=} 0, k \cancel{=}n$.
 
 
 Поймем, что при $n = 2^m$ это так. Докажем по индукции: при $n = 1; n = 2$ - верно, все числа между 1-цами в 1-ой и 2-ой строке треугольника П. четны. 
 Предположим тогда, что ${2^A \choose k}$ кратно $2$, $k \cancel{=} 0, k \cancel{=}2^A$. Докажем, что и ${2^{A + 1} \choose k}$ кратно 2.
 
 Поделим множество из $2^{A + 1}$ элементов пополам. Посчитаем количество способов выбрать $k$ элементов суммарно из двух наборов (как в задаче \textbf{3б}): из первого можно выбрать 0, а из второго - $k$ и т.д. (причем неважно, что $k$ может быть больше $2^A$, ведь ${2^{A + 1} \choose k} = {2^{A + 1} \choose 2^{A + 1} - k}$) Т.е всего:
 $$ \sum_{i=0}^{2 ^ A} {2^A \choose i} \cdot {2^A \choose 2^A - i}$$
 Но заметим, что по предположению индукции каждое число в сумме - есть произведение двух четных, кроме двух слагаемых (первого и последнего, которые равны единице), сумма которых тоже четна $\Rightarrow$ ${2^(A+1) \choose k}$ четно, при $k \cancel{=} 0, k\cancel{=}2^{A+1}$, значит все строки с номерами, равными степеням двойки состоят из четных чисел (кроме первого и последнего).
 
 Из построения треугольника следует, что такие "четные"\space строки могут быть получены только из "нечетных" (${n \choose k} = {n - 1 \choose k} + {n - 1 \choose k - 1}$, ${n - 1 \choose 0} = 1 \Rightarrow $ ${n - 1 \choose 1}$ не кратно 2 и т.д.) Значит каждая строка с номером $2^m - 1$ состоит из нечетных чисел.
 
 Осталось доказать, что не бывает иных четных строк (т.е. четных строк, номера которых не степень 2-ки). Предположим противное, и такая строка есть. Поймем, что если строка четна, то у нее четный номер ( ${n \choose 1} = n $ - обязательный член любой строки ). Тогда повторим прием из задачи \textbf{3б}: поделим $2n$-элементное множество пополам. Тогда число способов выбрать $k$ элементов:
			 $$\sum_{i=0}^{n} {n \choose k}^2$$
 Но по предположению это число четно, значит, четны все числа вида  $ {n \choose k} $. Тогда есть четная строка с номером $n$. Если $n=2a$, повторим до тех пор пока не получим, что $n' = 2b - 1$, и строка с этим номером четна. Такое возможно лишь, если $n' = 1$. Но тогда $2n = 2^x$ - противоречие $\Rightarrow$ иных четных строк отличных от описанных нет.
 \end{document}  