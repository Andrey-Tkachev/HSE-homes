\documentclass{article}
\usepackage{cancel}
\usepackage[utf8]{inputenc}
\usepackage[english,russian]{babel}
\begin{document}

\begin{center}
	\subsection*{Задача 1}
\end{center}
Представим пятизначное число $ a_4 .. a_0 $ в виде:
	 $$ a_4 * 10^4 + a_3 * 10^3 + .. + a_0 $$
Тогда сумма всех таких чисел с нечетными цифрами:
 $$ 10^4 * (a_{40} + a_{41} + .. + a_{44}) * k + .. + 10^0 * (a_{00} + a_{01} + .. + a_{04}) * k $$ 
Где $ k $ - количество таких чисел, у которых в $ i $ разряде стоит цифра $ a_{ij} $, $ j $ номер нечетной цифры.  

Поймем, что $ k = 5^4 $. И правда, для каждой из 5 нечетных цифр разряда $ i $ существует $ 5^3 $ вариантов других цифр из прочих разрядов, т.к. числа пятизначные. Поймем, что $ \sum_{k=0}^{4} a_{ik} = 1 + 3 + .. 9 = 25 $, где $ a_{ij} $  - нечетная цифра.
Получим, что тогда сумма таких пятизначных:
$$ 5^4 * 25 * (10 ^ 4 + .. + 10 ^ 0) = 5 ^ 6 * 11111 $$
\begin{center}
	\subsection*{Задача 2}
\end{center} 
\textbf{a)} Количество чисел от $ 1 $ до $ 10^6 $ в записи которых нет 1 равно $ 9 ^ 6 $.
Заметим, что $ 9 ^ 6 = 531441 > 10^6 : 2 $. $\Rightarrow$ Чисел без единицы больше половины среди первого миллиона, а значит их больше чем тех, в которых она есть.
\\
\textbf{b)} Количество чисел от $ 1 $ до $ 10^7 $, в записи которых нет 1 равно $ 9 ^ 7 $.
$$ 9 ^ 7  =  47829669 < 5000000 = 10^7 : 2 $$
\begin{center}
	Значит чисел без единицы в записи меньше.
\end{center}

\begin{center}
	\subsection*{Задача 3}
\end{center} 
$ \Omega $ - "Все десятизначные числа."
$$ |\Omega| = 10^{10} - 10^9 = 9 * 10^9 $$
Пусть $A = $ "В десятизначном числе есть хотя бы две одинаковые цифры."\\
Тогда посчитаем $ |\bar{A}| $. $ |\bar{A}| = 9 * 9 * 8 * ... * 1 = 9! $ (Первую цифру можно выбрать 9-ю способами, ибо начинаться с 0 десятизначное число не может, вторую 9-ю - т.к. тут ноль уже может стоять, третью 9-ю и т.д.). Т.к. $A + \bar{A} = \Omega$ (В десятизначном числе либо есть повторяющаяся цифра, либо ее нет):
$$ |A| = |\Omega| - |\bar{A}|$$
Тогда: $$ P(A) = \frac{|\Omega| - |\bar{A}|}{|\Omega|} = \frac{10^9 - 9!}{10^9} $$

\newpage

\begin{center}
	\subsection*{Задача 4}
\end{center} 

Возьмем человека и выберем ему пару. Сделать это можно $ 2 * n - 1 $ способом (с собой быть в паре нельзя). Осталось 2n - 2 людей.
Возьмем кого-нибудь и выберем ему пару $ 2n - 3 $ способами. Будем продолжать пока не останутся два человека, которых 1-м способом разобьем на пару (Обязательно останутся 2-ое, т.к. людей $ 2n $, и каждый раз мы берем 2-х людей). Тогда по правилу произведения число разбиений на пары:
$$ (2n - 1) * (2n - 3) * ... * (2n - (2i + 1)) * ... * (2n - (2n - 1)) = (2n - 1) * (2n - 3) * ...  * 1 $$
Нечетность этого числа следует из нечетности всех его множителей. 
\\
\begin{center}
	\subsection*{Задача 5}
\end{center} 

$ \Omega $ - "Все возможные колоды карт."
$$ |\Omega| = 52! $$
Пусть $A = $ "В колоде все карты одной масти расположены в порядке старшинства."

Порядок старшинства для каждой масти задается однозначно $ \Rightarrow $ количество исходов  $A$ определяется лишь числом вариантов взаимного расположения мастей в колоде. 
$$  |A| = {52 \choose 13} * {39 \choose 13} * {26 \choose 13} * {13 \choose 13} $$
(Число способов выбрать место масти в колоде определяется числом способов выбрать 13 свободных мест в колоде)\\Таким образом:
$$ P(A) = \frac{|A|}{|\Omega|} = \frac{1}{(13!) ^ 4}$$ 

\begin{center}
	\subsection*{Задача 6}
\end{center} 

\textbf{A)} $ \Omega $ - "Все наборы из 6 карт, взятых из колоды размером 32."
$$ |\Omega| = {32 \choose 6} $$
Пусть $A = $ "В наборе есть хотя бы один туз."

Тогда посчитаем $|\bar{A}|$, как количество способов набрать 6 карт из колоды без тузов:
$$|\bar{A}| = {32 - 4 \choose 6} = {28 \choose 6}$$
Заметим, что $ A + \bar{A} = \Omega $ (Наборы без тузов и наборы с тузами образуют все возможные наборы).Таким образом:
$$ P(A) = \frac{|\Omega| - |\bar{A}|}{|\Omega|} = \frac{{32 \choose 6} - {28 \choose 6}}{{32 \choose 6}}$$ 

\textbf{B)} $ \Omega $ - "Все наборы из 6 карт."
$$ |\Omega| = {32 \choose 6} $$
Пусть $A = $ "В наборе есть по карте каждой масти."
Посчитаем количество исходов при котором в наборе из 6 есть:
\\
 \begin{itemize} 
\item{$B$ - "3 карты одной масти, и все остальные - оставшихся мастей"}
\item{$C$ - "По две карты каких-то двух мастей и по одной оставшихся"}
\end{itemize} 
Заметим, что $ A = B + C $ (Больше 4-х карт одной масти в наборах из $A$ быть не может, иначе одной из мастей бы не было).
\\
\begin{flushleft}
	$|B| = {4 \choose 1} * {13 \choose 3} * 13^3$ (Выбрать какой масти будут 3 карты, и выбрать эти три карты, и выбрать по одной из оставшихся) 
\end{flushleft}
\begin{center}
	$|C| = {13 \choose 2}^4 $ (Выбрать 2 карты каждой масти)
\end{center}
Тогда:
$$ P(A) = \frac{|A|}{|\Omega|} = \frac{{4 \choose 1} * {13 \choose 3} * 13^3 +  {13 \choose 2}^4}{{32 \choose 6}}$$   

\begin{center}
	\subsection*{Задача 7}
\end{center} 

Заселить 4-х местную комнату можно $ {7 \choose 4} $ способами. Для каждого из них существует  $ {3 \choose 2} $ способов заселить 2-х местную.
Оставшуюся 1-о местную можно заполнить 1 способом.
Итого, количество способов заселить квартиру:
$$ {7 \choose 4} *  {3 \choose 2} $$
\\
\begin{center}
	\subsection*{Задача 8}
\end{center}

Выберем команды, которые будут хозяевами поля (половина из всего числа команд). Сделать это можно $ {16 \choose 8} $ способами. Затем для каждого хозяина выберем соперника из оставшихся команд. Сделать это можно $ 8 * 7 ... * 1 = 8! $ способами.
Тогда количество вариантов расписания:
$$ {16 \choose 8} * 8! = \frac{16!}{8!}$$ 

\begin{center}
	\subsection*{Задача 9}
\end{center}

Пронумеруем книги, и рассмотрим какую-нибудь их расстановку (одну из $20!$). Попробуем разбить эту расстановку на промежутки так, чтобы книги из одного промежутка оказались на одной полке, а самих промежутков было $5$ . Для этого нужно расставить $5 - 1 = 4$ перегородку между книгами  (Если пронумеровать перегородки от 0 до 4, то все, что стоит левее 0 перегородки, отправим на первую полку, все что между 0 и 1 - на вторую и т.д. Если между перегородками ничего нет, то это означает, что соответствующая полка пуста). Тогда представим, мы хотим как-то поставить 20 книг и 4 перегородки в пространстве (причем порядок перегородок, как и книг должен быть одним и тем же). Для этого разместим в 24-х ячейках 20 книг (выберем 20 ячеек и поставим в них книги в их порядке), а в оставшиеся поставим перегородки. Сделать это можно столькими способами: 
$$ {20 + 4 \choose 20} = {24 \choose 4}  $$
Тогда, для каждой расстановки 20 книг существует $ {24 \choose 4}  $ вариантов расставить их по 5 полкам с сохранением порядка. Тогда, все число способов расставить книги:
$$ 20! * {24 \choose 4}  $$

\end{document}

